% allgem. Dokumentenformat
\documentclass[a4paper,12pt,headsepline]{scrartcl}
%Variablen welche innerhalb der gesamten Arbeit zur Verfügung stehen sollen
\newcommand{\titleDocument}{Bachelor Thesis}
\newcommand{\subjectDocument}{in Informatics}


\newcommand{\specialcell}[2][c]{%
	\begin{tabular}[#1]{c}
		#2
	\end{tabular}
}

\newcommand{\specialcellleft}[2][@{}l]{%
	\begin{tabular}[#1]{@{}l}
		#2
	\end{tabular}
}

\newcommand{\fixme}[1]{
	~\\
	\noindent
	\textbf{\textcolor{red}{FIXME: #1}}
	\\
}




% weitere Pakete
% Grafiken aus PNG Dateien einbinden
\usepackage{graphicx}

% Eurozeichen einbinden
\usepackage[right]{eurosym}

% Umlaute unter UTF8 nutzen
\usepackage[utf8]{inputenc}

% Zeichenencoding
\usepackage[T1]{fontenc}

\usepackage{lmodern}
\usepackage{fix-cm}

\usepackage{svg}

% floatende Bilder ermöglichen
%\usepackage{floatflt}

% mehrseitige Tabellen ermöglichen
\usepackage{longtable}

% Unterstützung für Schriftarten
%\newcommand{\changefont}[3]{ 
%\fontfamily{#1} \fontseries{#2} \fontshape{#3} \selectfont}

\setcounter{secnumdepth}{4}
\setcounter{tocdepth}{4}

% Packet für Seitenrandabständex und Einstellung für Seitenränder
\usepackage{geometry}
\geometry{left=3.5cm, right=2cm, top=2.5cm, bottom=2cm}

% Paket für Boxen im Text
\usepackage{fancybox}

% bricht lange URLs "schoen" um
\usepackage[hyphens,obeyspaces,spaces]{url}

% Paket für Textfarben
\usepackage{color}

% Mathematische Symbole importieren
\usepackage{amssymb}

% auf jeder Seite eine Überschrift (alt, zentriert)
%\pagestyle{headings}

% erzeugt Inhaltsverzeichnis mit Querverweisen zu den Kapiteln (PDF Version)
\usepackage[bookmarksnumbered,pdftitle={\titleDocument},hyperfootnotes=false]{hyperref} 
%\hypersetup{colorlinks, citecolor=red, linkcolor=blue, urlcolor=black}
%\hypersetup{colorlinks, citecolor=black, linkcolor= black, urlcolor=black}

% neue Kopfzeilen mit fancypaket
\usepackage{fancyhdr} %Paket laden
\pagestyle{fancy} %eigener Seitenstil
\fancyhf{} %alle Kopf- und Fußzeilenfelder bereinigen
\fancyhead[L]{\nouppercase{\leftmark}} %Kopfzeile links
\fancyhead[C]{} %zentrierte Kopfzeile
\fancyhead[R]{\thepage} %Kopfzeile rechts
\renewcommand{\headrulewidth}{0.4pt} %obere Trennlinie
%\fancyfoot[C]{\thepage} %Seitennummer
%\renewcommand{\footrulewidth}{0.4pt} %untere Trennlinie

% für Tabellen
\usepackage{array}

% Runde Klammern für Zitate
%\usepackage[numbers,round]{natbib}

% Festlegung Art der Zitierung - Havardmethode: Abkuerzung Autor + Jahr
\bibliographystyle{alphadin}

% Schaltet den zusätzlichen Zwischenraum ab, den LaTeX normalerweise nach einem Satzzeichen einfügt.
\frenchspacing

% Paket für Zeilenabstand
\usepackage{setspace}

% für Bildbezeichner
\usepackage{capt-of}

% für Stichwortverzeichnis
\usepackage{makeidx}

% für Listings
\usepackage{listings}
\lstset{numbers=left, numberstyle=\tiny, numbersep=5pt, keywordstyle=\color{black}\bfseries, stringstyle=\ttfamily,showstringspaces=false,basicstyle=\footnotesize,captionpos=b}
\lstset{language=java}

% Indexerstellung
\makeindex

% Abkürzungsverzeichnis
\usepackage[english]{nomencl}
\let\abbrev\nomenclature

% Abkürzungsverzeichnis LiveTex Version
\renewcommand{\nomname}{Abbreviations}
\setlength{\nomlabelwidth}{.25\hsize}
\renewcommand{\nomlabel}[1]{#1 \dotfill}
\setlength{\nomitemsep}{-\parsep}
\makenomenclature
%\makeglossary

% Abkürzungsverzeichnis TeTEX Version
% \usepackage[german]{nomencl}
% \makenomenclature
% %\makeglossary
% \renewcommand{\nomname}{Abkürzungsverzeichnis}
% \setlength{\nomlabelwidth}{.25\hsize}
% \renewcommand{\nomlabel}[1]{#1 \dotfill}
% \setlength{\nomitemsep}{-\parsep}

% Disable single lines at the start of a paragraph (Schusterjungen)
\clubpenalty = 10000
% Disable single lines at the end of a paragraph (Hurenkinder)
\widowpenalty = 10000
\displaywidowpenalty = 10000

\begin{document}
% hier werden die Trennvorschläge inkludiert
\input{latex_settings/trennung}

%Schriftart Helvetica
%\changefont{phv}{m}{n}

% Leere Seite am Anfang
\newpage
\thispagestyle{empty} % erzeugt Seite ohne Kopf- / Fusszeile

% Titelseite %
% das Papierformat zuerst
%\documentclass[a4paper, 11pt]{article}

% deutsche Silbentrennung
%\usepackage[ngerman]{babel}

% wegen deutschen Umlauten
%\usepackage[ansinew]{inputenc}

% hier beginnt das Dokument
%\begin{document}


\thispagestyle{empty}

%\begin{figure}[t]
% \includegraphics[width=0.6\textwidth]{abb/fh_koeln_logo}
%\end{figure}

\begin{figure}[t]
 \centering
 \includegraphics[width=0.6\textwidth]{abb/logo1}
~~~~~~~~~~
 \includegraphics[width=0.3\textwidth]{abb/logo2}
\end{figure}


\begin{verbatim}


\end{verbatim}

\begin{center}
\Large{Universität Bayreuth}\\
\end{center}


\begin{center}
\Large{Fakultät für Informatik}
\end{center}
\begin{verbatim}




\end{verbatim}
\begin{center}
\doublespacing
\textbf{\LARGE{\titleDocument}}\\
\singlespacing
\begin{verbatim}

\end{verbatim}
\textbf{{~\subjectDocument}}
\end{center}
\begin{verbatim}

\end{verbatim}
\begin{center}

\end{center}
\begin{verbatim}

\end{verbatim}
\begin{center}
\textbf{zur Erlangung des akademischen Grades \\ Bachelor / Master of Science}
\end{center}
\begin{verbatim}






\end{verbatim}
\begin{flushleft}
\begin{tabular}{llll}
\textbf{Thema:} & & Integration of JPA-conform ORM-Implementations in Hibernate Search & \\
& & \\
\textbf{Autor:} & & Martin Braun <martinbraun123@aol.com>& \\
& & Matrikel-Nr. 1249080 & \\
& & \\
\textbf{Version vom:} & & \today &\\
& & \\
\textbf{1. Betreuer:} & & Dr. Bernhard Volz &\\
\textbf{2. Betreuer:} & & Prof. Dr. Bernhard Westfechtel &\\
\end{tabular}
\end{flushleft}

% römische Numerierung
%\pagenumbering{arabic}

% 1.5 facher Zeilenabstand
\onehalfspacing

% Einleitung / Abstract
\section*{Zusammenfassung}
Hibernate Search

%\begin{verbatim}

%

%\end{verbatim}

\section*{Abstract}


% einfacher Zeilenabstand
\singlespacing

% Inhaltsverzeichnis anzeigen
\newpage
\tableofcontents

% das Abbildungsverzeichnis
%\newpage
% Abbildungsverzeichnis soll im Inhaltsverzeichnis auftauchen
%\addcontentsline{toc}{section}{List of figures}
% Abbildungsverzeichnis endgueltig anzeigen
%\listoffigures

% das Tabellenverzeichnis
%\newpage
% Abbildungsverzeichnis soll im Inhaltsverzeichnis auftauchen
%\addcontentsline{toc}{section}{Tabellenverzeichnis}
% \fancyhead[L]{Abbildungsverzeichnis / Abkürzungsverzeichnis} %Kopfzeile links
% Abbildungsverzeichnis endgueltig anzeigen
%\listoftables

%% WORKAROUND für Listings
%\makeatletter% --> De-TeX-FAQ
%\renewcommand*{\lstlistoflistings}{%
%  \begingroup
%    \if@twocolumn
%      \@restonecoltrue\onecolumn
%    \else
%      \@restonecolfalse
%    \fi
%    \lol@heading
%    \setlength{\parskip}{\z@}%
%    \setlength{\parindent}{\z@}%
%    \setlength{\parfillskip}{\z@ \@plus 1fil}%
%    \@starttoc{lol}%
%    \if@restonecol\twocolumn\fi
%  \endgroup
%}
%\makeatother% --> \makeatletter
% das Listingverzeichnis
%\newpage
% Listingverzeichnis soll im Inhaltsverzeichnis auftauchen
%\addcontentsline{toc}{section}{Listingverzeichnis}
%\fancyhead[L]{Abbildungs- / Tabellen- / Listingverzeichnis} %Kopfzeile links
%\renewcommand{\lstlistlistingname}{Listingverzeichnis}
%\lstlistoflistings
%%%%

% das Abkürzungsverzeichnis
%\newpage
% Abkürzungsverzeichnis soll im Inhaltsverzeichnis auftauchen
%\addcontentsline{toc}{section}{Abkürzungsverzeichnis}
% das Abkürzungsverzeichnis entgültige Ausgeben
%\fancyhead[L]{Abkürzungsverzeichnis} %Kopfzeile links
%\input{latex_settings/abbreviations/abbreviations}
%\printnomenclature

% Definiert Stegbreite bei zweispaltigem Layout
\setlength{\columnsep}{25pt}

%%%%%%% EINLEITUNG %%%%%%%%%%%%
%\twocolumn
\newpage
\fancyhead[L]{\nouppercase{\leftmark}} %Kopfzeile links

% 1,5 facher Zeilenabstand
\onehalfspacing










% einzelne Kapitel
% !TeX spellcheck = en_GB
\section{Preface}\label{Preface}
In the software world, or more specific, the Java enterprise world, developers tend to abstract access to data in a way that components are interchangeable. A perfect example for such an abstraction is the usage of Object Relational Mappers (ORM). The database specifics are mostly irrelevant to the average developer and the need for native SQL is brought down to a minimum. This makes the switch to a different relational database system (RDBMS) easier in the later stages of a product's life cycle.
\\\\
The Java Persistence API (JPA) went even further by standardising ORMs. First conceived in 2006 \cite{needed}, it is now the de-facto standard for Object Relational Mappers in Java. The developer doesn't need to know which specific ORM is used in the application, as all the database queries are written against a standardized query API and therefore portable. This means that not only the database is interchangeable, but even the specific ORM, it is accessed by, is as well.
\\\\
However, this does not mean that all JPA implementations ship with the same features. While all of them are JPA compliant (apart from minor bugs), some ship with additional modules to enhance their capabilities. A perfect example for this is the Hibernate Search API aimed at Hibernate ORM users: Nowadays, even small applications like online shops need enhanced search capabilities to let the user find more results for a given input.
\\\\
This is not something a regular RDBMS excels at and Hibernate Search comes into use: It works atop the Hibernate ORM/JPA system and enables the developer to index the domain model for searching. It's not only a mapper from JPA entities to a search index, but also keeps the index up-to-date if something in the database changes.
\\\\
\begin{figure}[ht]
	\centering
	\includegraphics[scale=0.5]{images/hibernate_search_hibernate_schema.pdf}
		\caption{Hibernate Search with Hibernate ORM}
		\label{fig1}
\end{figure}
\\
Hibernate Search, which is based on the powerful Lucene search toolbox, is a separate project in the Hibernate family and is using a lot of JPA interfaces in its codebase and aims to provide a JPA "feeling" in its API. However, this does not mean that it is compatible with other JPA providers than Hibernate ORM (apart from Hibernate OGM, the NoSQL JPA mapper of the family).
\\\\
\begin{figure}[ht]
	\centering
	\includegraphics[scale=0.5]{images/hibernate_search_any_jpa_problem_schema.pdf}
	\caption{Hibernate Search's incompatibility with other JPA implementations}
	\label{fig2}
\end{figure}
\\
While using Hibernate Search obviously is beneficial for Hibernate ORM applications, not all developers can bind themselves to a specific JPA implementation in their application. For some, the ability to change the it is of strategic important, for others it is just sheer preference to use a different JPA implementation.
\\\\
Currently these developers have to resort to using different full text search systems like native Lucene, ElasticSearch or Solr. While this is always a viable option, for some applications Hibernate Search would be a much better suit because of it's design with a entity structure in mind and the automatic index updating feature, if it just were compatible with generic JPA.
\\\\
When investigating Hibernate Search's project structure \cite{source-code-git}, we see that the only module apart from some server-integration modules that depends on any ORM logic is "hibernate-search-orm". The modules that contain the indexing engine, the replication logic, alternative backends, etc. are completely independent from any ORM logic. This means, that we can reuse most of the codebase for a generic version of Hibernate Search.
\\\\
In this thesis we will show how such a generic version can be built. We will look at how Hibernate Search's engine can be reused. Then, we will write a standalone version of this engine and finally integrate it with generic JPA.

\pagebreak

\section{Overview}\label{Introduction}

\subsection{An overview of different database paradigms}

\textcolor{red}{das hier anders}

When it comes to persisting data in applications, nowadays there exist a lot of different
paradigms that one has to choose from. Following is a short explanation for the two currently most used ones.

\subsubsection{relational databases}

\subsubsection{NoSQL databases}

\subsection{Object Relational Mappers}

While the NoSQL approach is undeniably rising in popularity, the demand for
relational solutions is still unmatched, as it has been used and has proven itself in practice
for many years now.
\\\\
Nowadays, many popular languages like Java, C\#, etc. are object-oriented.
While SQL solutions for querying relational databases exist for these languages, the user either has to work with the rowsets manually or convert them into custom data access objects. Both approaches include a lot of manual work.
\\\\
This is where Object Relational Mappers (ORM) come into use. They map tables to entity-classes and
enable users to write queries against these classes instead of tables. This is especially useful if used
in big software products as not all programmers have to know the exact details of the underlying database. The database system could even be completely replaced for another (provided the ORM supports the specific RDBMS), with the business logic not changing a bit.

\subsubsection{JPA}

The first version of the JPA standard was released in May 2006. From then on it rose to probably the most commonly used persistence API for Java. While mostly known for standardizing relational database mappers (ORM), it supports other concepts like NoSQL or XML storage as well. However, when talking about JPA in this thesis we will be focusing on the relational aspects of it. Currently, the newest version of this standard is 2.1.\footnote{Wikipedia on Java Persistence API, see~\cite{wiki_jpa}}.
\\\\
Some popular relational implementations are:
\begin{itemize}
	\item Hibernate ORM (JBoss)
	\item EclipseLink (Eclipse foundation)
	\item OpenJPA (Apache foundation)
\end{itemize}
\textcolor{red}{vll Beispiel für eine einfache Beziehung, ER-Modell vs. gemappte Klasse}
\\\\
Using the standardized JPA API over any native ORM API has one really interesting benefit:
The specific JPA implementation can be swapped out. This is particularily important if you are working in a Java EE environment. Java EE itself is a specification for platforms, mostly Web-servers (JPA is part of the Java EE spec).\footnote{Wikipedia on Java EE, see~\cite{wiki_java_ee}} Many Java EE Web-servers ship with a bundled JPA implementation that they are optimized for. This means that if a user switches servers, he/she is also likely to swap out the JPA implementor. If everything in the application is written in a JPA compliant way, the user will then generally not run into many problems related to this switch.

\subsection{Fulltext search}

Conventional relational databases are good at retrieving and querying structured data. But if one wants to build a search engine atop a domain model, most RDBMS will only support the SQL-LIKE operator:\\

\lstset{language=sql}
\begin{lstlisting}[frame=htrbl, label={lst:result2}]
SELECT book.id FROM book WHERE book.name LIKE %name%;
\end{lstlisting}
While this might be enough for some applications, this wildcard query doesn't support features a good search engine would need, for example:

\begin{itemize}
	\item fuzzy queries (variations of the original string will get matched, too)
	\item phrase queries (search for a specified phrase)
	\item regular expression queries (matches are determined by a regular expression)
\end{itemize}
There may exist some RDBMS that support similar query-types, but in the context of using a ORM we would then lose the ability to switch databases since we require specific features not every RDBMS supports.
\\\\
Fulltext search engines can be used to complement databases in this regard. They are not intended to be replacing the database, but to add additional functionality by indexing the data that is to be searched in a more sophisticated way. We will now take a look at some of the most popular available options for Java developers while focusing on their usage, features, the pros and cons of using them, and compatibility with the JPA standard.

\subsubsection{Lucene}

\textcolor{red}{mention current version for each of these?}

\begin{quote}
Apache LuceneTM is a high-performance, full-featured text search engine library written entirely in Java. It is a technology suitable for nearly any application that requires full-text search, especially cross-platform.\footnote{official Lucene website, see~\cite{lucene_apache_org}}
\end{quote}
Lucene serves as the basis for most fulltext search engines written in Java. It has many different utilties and modules aimed at search engine developers. However, it can be used on its own as well.

\paragraph{Index structure}
Lucene uses an \textbf{inverted index} to store data. This means that instead of storing texts mapped to the words contained in them, it works the other way around. All different words (or terms) are mapped to the texts they occur in.\footnote{Lucene basic concepts, see~\cite{lucene_basic_concepts}} Also, before anything can be searched using Lucene, it has to be added to the the index first.

\paragraph{Concepts} Lucene has its own set of concepts that need to be discussed first before we can take a look at it's usage. Following is the explanation of the most important ones.

\subparagraph{Documents}
Documents are the data-structure Lucene stores and retrieves from the index. A index can contain zero or more Documents. Documents are added to the index with an IndexWriter and retrieved via an IndexReader/IndexSearcher.

\subparagraph{Fields}
A Document consists of at least one field. Fields are basically tuples of key and value. They can be stored (can be retrieved from the index) and/or indexed (can be searched on).

\subparagraph{Analyzers}
Before documents get indexed, their fields are analyzed first. Analysis is the process of modifying the input in a manner such that it can be searched upon (stemming, tokenization, ...). In Lucene this is done by special classes called Analyzers.

\paragraph{Usage - Indexing}
We will now take a look at how data is indexed in Lucene. In the following example we consider the data to be already present in form of a List of objects of the class 'Text' and concentrate on the Lucene usage itself.
\\\\
\textcolor{red}{Beispiel für Lucene usage hier, }

\paragraph{Usage - Searching}
Searching\\\\
\textcolor{red}{Beispiel für Lucene usage hier, }

\paragraph{Features}
Lucene is probably the most complete toolbox to build a search-engine from.

\paragraph{Pros and Cons}

\paragraph{Compatibility with JPA}
By design, Lucene out of the box is not very compatible with the JPA standard. For one, the flat document structure forces the user to de-normalize the entity model before indexing. Secondly, since every search-relevant change in the database should be reflected in the index, it must be kept up to date. When using Lucene, this has to be done completely manually as it natively doesn't have any integration with databases.

\subsubsection{Solr}

\subsubsection{ElasticSearch}

\subsubsection{Hibernate Search}

\textcolor{red}{some kind of conclusion with a table of features. -> Hibernate Search, aber mit dem Problem von Kompatibilität mit Non Hibernate ORM, mention Compass?}

\subsection{aims of this thesis}


% !TeX spellcheck = en_GB
\section{Overview of technologies}\label{Overview}
Before we start going into detail about how to work with Hibernate Search in a generic environment, we will give a short overview of relevant technologies first. We will explain why ORMs in general and the JPA specification in particular are beneficial. Then, we will explain what fulltext search engines are used for and give a short overview about the available solutions for Java. We will see that generalizing Hibernate Search for any JPA implementation is a good approach and that it has benefits over using the different search solutions available.

\subsection{Object Relational Mappers}
Nowadays, many popular languages like Java, C\#, etc. are object-oriented\footnote{Wikipedia on Object Oriented Programming (OOP), see~\cite{object_oriented_programming_wiki}}.
While SQL solutions for querying relational databases exist for these languages (JDBC for Java\footnote{Oracle JDBC overview, see~\cite{jdbc_oracle}}, OleDb for C\#\footnote{OleDb usage page, see~\cite{oledb_ms}}), the user either has to work with the rowsets manually or convert them into custom data transfer objects (DTO) to gain at least some "real" objects to work with. Both approaches don't suit the object oriented paradigm well as SQL "flattens" the data into rows with when querying while a well designed class model would work with multiple classes in a hierarchy.
\\
\lstset{language=sql}
\begin{lstlisting}[frame=htrbl, caption={sql query "flattening" the author and book table into rows}, label={lst:flattening.sql}]
SELECT author.id, author.name, book.id, book.name 
FROM author_book, author, author
WHERE author_book.bookid = book.id
AND author_book.authorid = author.id
\end{lstlisting}
~\\
This is where Object Relational Mappers (ORM) come into use. They map tables to entity-classes and
enable users to write queries against these classes instead of tables. The returned objects are part of a complex object hierarchy and are easier to use from a object oriented point of view.
\\
\lstset{language=java}
\begin{lstlisting}[frame=htrbl, caption={ORM query example}, label={lst:flattening.sql}]
List<Author> data = orm.query("SELECT a FROM Author a " +
	"LEFT OUTER JOIN a.books");
for(Author author : data) {
	System.out.println("name: " + author.getName() + 
		", books: " + author.getBooks());
}
\end{lstlisting}
~\\
This is especially useful if used in big software products as not all programmers have to know the exact details of the underlying database. The database system could even be completely replaced for another (provided the ORM supports the specific RDBMS), while the business logic would not changing a bit.

\subsection{JPA}

The first version of the JPA standard was released in May 2006. From then on it rose to being probably the most commonly used persistence API for Java and is considered the "industry standard approach for Object Relational Mapping"\footnote{Wikibooks on Java Persistence, see~\cite{wikibooks_on_jpa}}. While mostly known for standardizing relational database mappers (ORM), it also supports other concepts like NoSQL\footnote{Hibernate OGM project homepage, see~\cite{hibernate_ogm}} \footnote{EclipseLink project homepage, see~\cite{eclipselink}} or XML storage\footnote{EclipseLink project homepage, see~\cite{eclipselink}}. However, when talking about JPA in this thesis, we will be focusing on the relational aspects of it. Currently, the newest version of this standard is 2.1.\footnote{Wikipedia on Java Persistence API, see~\cite{wiki_jpa}}.
\\\\
Some popular relational implementations are:
\begin{itemize}
	\item Hibernate ORM (JBoss)\footnote{Hibernate ORM project homepage, see~\cite{hibernate_orm}}
	\item EclipseLink (Eclipse foundation)\footnote{EclipseLink project homepage, see~\cite{eclipselink}}
	\item OpenJPA (Apache foundation)\footnote{OpenJPA project homepage, see~\cite{openjpa}}
\end{itemize}
~\\
Using the standardized JPA API over any native ORM API has one really interesting benefit:
The specific JPA implementation can be swapped out as it comes with standards for many common use cases.
\\\\
This is particularily important if you are working in a Java EE environment. Java EE itself is a specification for platforms, mostly Web-servers (JPA is part of the Java EE spec).\footnote{Wikipedia on Java EE, see~\cite{wiki_java_ee}} Many Java EE Web-servers ship with a bundled JPA implementation that they are optimized for (Wildfly with Hibernate ORM, GlassFish with EclipseLink, ...). This means that if the server is switched, it could also be a reasonable idea to swap out the JPA implementor. If everything in the application is written in a JPA compliant way, the user will then generally not run into many problems related to this switch.

\pagebreak

\subsection{Fulltext search engines}

Conventional relational databases are good at retrieving and querying structured data. But if one wants to build a search engine atop a domain model, most RDBMS will only support the SQL-LIKE operator \footnote{w3schools on SQL LIKE, see~\cite{sql_like_w3schools}}:\\

\lstset{language=sql}
\begin{lstlisting}[frame=htrbl, caption={SQL LIKE operator in use}, label={lst:result2}]
SELECT book.id, book.name FROM book WHERE book.name LIKE %name%;
\end{lstlisting}
While this might be enough for some applications, this wildcard query doesn't support features a good search engine would need, for example:

\begin{itemize}
	\item fuzzy queries (variations of the original string will get matched, too)
	\item phrase queries (search for a specified phrase)
	\item regular expression queries (matches are determined by a regular expression)
\end{itemize}
There may exist some RDBMS that support similar query-types, but in the context of using a ORM we would then lose the ability to switch databases since, we would use vendor-specific features not every RDBMS supports.
\\\\
Fulltext search engines can be used to complement databases in this regard. They are generally not intended to be replacing the database, but add additional functionality by indexing the data that is to be searched in a more sophisticated way. We will now take a look at some of the most popular available options for Java developers focusing on their usage, features and compatibility with the JPA standard.
\\\\
\textcolor{red}{Was bedeutet compatibility with JPA? Näher spezifizieren}

\pagebreak

\subsubsection{Lucene}

\textcolor{red}{mention current version for each of these?}

\begin{quote}
	Apache Lucene™ is a high-performance, full-featured text search engine library written entirely in Java. It is a technology suitable for nearly any application that requires full-text search, especially cross-platform.\footnote{official Lucene website, see~\cite{lucene_apache_org}}
\end{quote}
Lucene serves as the basis for many fulltext search engines written in Java. It has many different utilties and modules aimed at search engine developers. However, it can be used on its own as well.

\paragraph{Concepts}

As Lucene's focus is not on storing relational data, it comes with its own set of concepts. Following is a short overview over the concepts it has. These are not only the basis for Lucene, but also for the other search engines we will discuss later as they are based on Lucene's rich set of features.

\subparagraph{Index structure}
Lucene uses an \textbf{inverted index} to store data. This means that instead of storing texts mapped to the words contained in them, it works the other way around. All different words (terms) are mapped to the texts they occur in\footnote{Lucene basic concepts, see~\cite{lucene_basic_concepts}}, so it can be compared to a \(Map<String, List<Text>>\) in Java. Before anything can be searched using Lucene, it has to be added to the the index (indexed) first.

\subparagraph{Documents}
Documents are the data-structure Lucene stores and retrieves from the index. An index can contain zero or more Documents.

\subparagraph{Fields}
A Document consists of at least one field. Fields are basically tuples of key and value. They can be stored (retrievable from the index) and/or indexed (used for searches, generate hits).

\subparagraph{Analyzers}
Before documents get indexed, their fields are analyzed with one of the many Analyzers first. Analysis is the process of modifying the input in a manner such that it can be searched upon (stemming, tokenization, ...).

\paragraph{Usage}
Using Lucene as a standalone requires the programmer to design the engine from the bottom up. The developer has to write all the logic, starting with the actual index writing control mechanism, and the conversion code from Java objects to Documents, through to the index searcher control, and the query code with the conversion from Documents back to Java objects. This whole process requires a lot of code to be written and the API just helps by providing the necessary tools. This has one additional problem though: The Lucene API tends to change a lot between versions and the code has to be kept up-to-date. It's not uncommon that whole features that worked in one version are deprecated (potentially unstable, marked to be removed in the future) in the next release, resulting in big code changes being potentially necessary.

\paragraph{Features}
Lucene probably is the most complete toolbox to build a search-engine from. It has pre-built analyzers for many languages, a queryparser to support user written queries, a phonetic module, a faceting module, and many more features. First starting as a text only search engine, it has had support for spatial indexing for some while now. This and the support for faceting means that it is increasingly becoming more of a general search toolbox.

\paragraph{Compatibility with JPA}
By design, Lucene out of the box is not very compatible with the JPA standard. For one, the flat document structure forces the user to de-normalize the entity model before indexing as it doesn't even have a mapper to convert from objects to its document representation and back. Secondly, since every search-relevant change in the database should be reflected in the index, it must be kept up-to-date. When using Lucene however, this has to be done completely manually as it natively doesn't have any integration with databases.

\subsubsection{Solr}

\begin{quote}
	Solr is the popular, blazing-fast, open source enterprise search platform built on Apache Lucene™.
\end{quote}

\subsubsection{ElasticSearch}

\subsubsection{Hibernate Search}

\begin{quote}
Hibernate Search transparently indexes your objects and offers fast regular, full-text and geolocation search. Ease of use and easy clustering are core.\footnote{Hibernate Search project homepage, see~\cite{hibernate_search_homepage}}
\end{quote}

\textcolor{red}{some kind of conclusion with a table of features. -> Hibernate Search, aber mit dem Problem von Kompatibilität mit Non Hibernate ORM, mention Compass?}

\subsection{Reasoning of decision for Hibernate Search}

\pagebreak

% !TeX spellcheck = en_GB

~
\pagebreak

\section{Challenges}\label{Challenges}
While building the generic version of Hibernate Search, we will encounter some challenges. We will discuss the biggest ones after we have introduced a small example project first. This project will be used to showcase some problems and usages later on in this thesis as well.

\subsection{The example project} \label{example_project}
Consider a software built with JPA that is used to manage the inventory of a bookstore. It stores information about the available books (ISBN, title, genre, short summary of the contents) and the corresponding authors (surrogate id, first \& last name, country) in a relational database. Each author is related to zero or more Books and each Book is written by one or more Authors. The entity relationship model diagram defining the database looks like this:
\\
\begin{figure}[ht]
	\centering
	\includegraphics[scale = 0.9]{images/Sample_Project_ER.pdf}
	\caption{the bookstore entity relationship model}
	\label{fig3}
\end{figure}
\\
Using a mapping table for the M:N relationship of Author and Book, the database contains three tables: Author, Book and Author\_Book. The applications strictly uses JPA to access the data without any vendor specific features. The JPA annotated classes for these entities are defined as the following listings show.

\pagebreak

\lstset{language=java}
\begin{lstlisting}[frame=htrbl, caption={Book.java}, label={lst:book.java_1}]
@Entity
@Table(name = "Book")
public class Book {

	@Id
	@Column(name = "isbn")
	private String isbn;
	
	@Column(name = "title")
	private String title;
	
	@Column(name = "genre")
	private String genre;
	
	@Lob
	@Column(name = "summary")
	private String summary;
	
	@ManyToMany(mappedBy = "books", cascade = {
		CascadeType.MERGE,
		CascadeType.DETACH,
		CascadeType.PERSIST,
		CascadeType.REFRESH
	})
	private Set<Author> authors;
	
	//getters & setters ...
}
\end{lstlisting}

\pagebreak

\lstset{language=java}
\begin{lstlisting}[frame=htrbl, caption={Author.java}, label={lst:author.java_1}]
@Entity
@Table(name = "Author")
public class Author {
	
	@Id
	@GeneratedValue(strategy = GenerationType.AUTO)
	@Column(name = "authorId")
	private Long authorId;
	
	@Column(name = "firstName")
	private String firstName;
	
	@Column(name = "lastName")
	private String lastName;
	
	@Column(name = "country")
	private String country;
	
	@ManyToMany(cascade = {
		CascadeType.MERGE, 
		CascadeType.DETACH, 
		CascadeType.PERSIST, 
		CascadeType.REFRESH
	})
	@JoinTable(name = "Author_Book", 
		joinColumns = 
			@JoinColumn(name = "authorFk", 
				referencedColumnName = "authorId"),
		inverseJoinColumns = 
			@JoinColumn(name = "bookFk", 
				referencedColumnName = "isbn"))
	private Set<Book> books;
	
	//getters & setters ...
}
\end{lstlisting}
For the sake of simplicity and since every JPA provider is able to derive a default DDL script from the annotations, we don't supply any information about how to create the schema here. However, for real world applications defining a hand-written DDL script might be a better idea since the generated code might not be optimal and differs between the different JPA implementations and RDBMSs used.

\pagebreak

\subsection{Standalone version} \label{problem_indexing_searching}
Hibernate Search's engine wasn't designed to be used directly by application developers. Its main purpose is to serve as an integration point for other APIs that need to leverage its power to index object graphs and query the index for hits by exposing a quite low-level and in some ways complex API. This is why we have to write our own standalone version based on the "hibernate-search-engine" serving as an abstraction layer such that it eases the usage of the engine in our JPA integration.

\subsection{JPA integration}
After the standalone version is finished, we will build an integration of it with JPA. By incorporating the same engine that the original does, we will support the same indexing behaviour and even stay compatible with entities designed for the original with as little changes as possible. In fact the main goal for the JPA integration is to be as compatible as possible with Hibernate Search ORM.
\\\\
In general we are aiming for an architecture that looks similar to this:

\begin{figure}[ht]
	\centering
	\includegraphics[scale=0.5]{images/hibernate_search_genericjpa_complete_architecture.pdf}
	\caption{Complete Architecture of Hibernate Search GenericJPA}
	\label{hibernate_search_genericjpa_complete_architecture}
\end{figure}

\pagebreak

%\subsection{Index rebuilding}
%If the way objects are indexed changes, the existing files have to %be purged and recreated in the new index format. The naive approach %here would be purging the index and then indexing all data %sequentially as they are retrieved from the database:
%\\
%\lstset{language=java}
%\begin{lstlisting}[frame=htrbl, caption={naive index rebuilding}, %label={lst:naiveIndexing}]
%EntityManager em = ...;
%<Hibernate Search Controller> search = ...;
%
%search.purgeAll(Book.class);
%
%Query query = em.createQuery("SELECT b FROM Book b");
%List<Book> booksFromDb = query.getResultList();
%for(Book b : booksFromDb) {
%	search.index(b);
%}
%\end{lstlisting}
%While this might work for small databases, bigger datasets will %cause this algorithm to run out of memory, since we just retrieve %all the data at once. This could be fixed by implementing a %batching strategy, but it would still be quite slow as it only uses %one thread which would mostly be used for I/O from the database.
%\\\\
%This is not optimal, since a index rebuild should be as fast as %possible as the application cannot be properly used while the job %is running. Therefore we need to create a parallel indexing %mechanism, just like the one Hibernate Search ORM has.

\subsection{Automatic index updating} \label{automatic_indexing_problematic_intro}
The most important feature to be re-built, is automatic index updating. In Hibernate Search ORM, every change in the database is automatically reflected in the index. It is important to have this feature, because otherwise developers would have to manually make sure the index is always up-to-date. With bigger project sizes it gets increasingly harder to keep track of all the locations in the code that change index relevant data and inconsistencies in the indexing logic become nearly unavoidable. While this problem might be mitigated by hiding all the database access logic behind a service layer, even such a solution would be hard to keep error-free as for big applications this layer will probably have multiple critical indexing relevant spots as well.
\\\\
The original Hibernate Search ORM is achieving an up-to-date index by listening to specific Hibernate ORM events for all of the C\_UD (CREATE, UPDATE, DELETE) actions. These events also cover entity relationship collections (for example represented by mapping tables like Author\_Book). As our goal is to create a generic Hibernate Search engine that works with any JPA implementation, we cannot rely on any vendor specific event system. Thus, at least an additional generic solution has to be found.

\subsection{Timeline}
The solutions for the challenges depend on each other in the same order they were described above as the JPA integration can only be worked on as soon as the standalone integration is done and work on the automatic updating mechanism cannot be started without knowing the JPA integration interfaces. The timeline of our project therefore looks like this:
\\
\begin{figure}[ht]
	\centering
	\includegraphics[scale=0.75]{images/timeline_genericjpa_complete.pdf}
	\caption{Timeline of the project}
	\label{project_timeline}
\end{figure}

\pagebreak

% !TeX spellcheck = en_GB
\section{indexing \& searching}

In this section we will start by discussing how Hibernate Search's engine (in the form of the module "hibernate-search-engine") can be used in general. Then we will work out a standalone version of this engine that is easier to work with and lastly we will show how we integrate this standalone version with JPA.

\subsection{Setting up the example project}

Before we explain how we do things in particular, we set up the example entities described in \ref{example_project} as if the original Hibernate Search would have been used. We do so by adding additional annotations to our entity-classes:

\begin{enumerate}
	\item \textbf{@Indexed}: marks the entity as an index root-type.
	\item \textbf{@DocumentId}: marks the field as the id of this entity. this is only needed if no JPA @Id can be found, but can be used to override settings.
	\item \textbf{@Field}: describes how the annotated field should be indexed. The fieldname defaults to the property name.
	\item \textbf{@IndexedEmbedded}: marks properties that point to other classes which should be included in the index. By default, all fields contained in these entities are prefixed with the property name this is placed on.
	\item \textbf{@ContainedIn}: used in entities that are embedded in other indexes. this is set on the properties that point back to the index-owning entity.
\end{enumerate}
\noindent
The resulting entities look like this:
\\
\lstset{language=java}
\lstset{moredelim=[is][\bfseries]{[*}{*]}}
\begin{lstlisting}[frame=htrbl, caption={Book.java with Hibernate Search annotations}, label={lst:book.java_2}]
@Entity
@Table(name = "Book")
[*@Indexed*]
public class Book {

	@Id
	@Column(name = "isbn")
	[*@DocumentId*]
	private String isbn;
	
	@Column(name = "title")
	[*@Field(store = Store.YES, index = Index.YES)*]
	private String title;
	
	@Column(name = "genre")
	[*@Field(store = Store.YES, index = Index.YES)*]
	private String genre;
	
	@Lob
	@Column(name = "summary")
	[*@Field(store = Store.NO, index = Index.YES)*]
	private String summary;
	
	@ManyToMany(mappedBy = "books", cascade = {
		CascadeType.MERGE,
		CascadeType.DETACH,
		CascadeType.PERSIST,
		CascadeType.REFRESH
	})
	[*@IndexedEmbedded(includeEmbeddedObjectId = true)*]
	private Set<Author> authors;
	
	//getters & setters ...
}
\end{lstlisting}

\lstset{language=java}
\lstset{moredelim=[is][\bfseries]{[*}{*]}}
\begin{lstlisting}[frame=htrbl, caption={Author.java with Hibernate Search annotations}, label={lst:author.java_2}]
@Entity
@Table(name = "Author")
public class Author {

	@Id
	@GeneratedValue(strategy = GenerationType.AUTO)
	@Column(name = "authorId")
	[*@DocumentId*]
	private Long authorId;
	
	@Column(name = "firstName")
	[*@Field(store = Store.YES, index = Index.YES)*]
	private String firstName;
	
	@Column(name = "lastName")
	[*@Field(store = Store.YES, index = Index.YES)*]
	private String lastName;
	
	@Column(name = "country")
	[*@Field(store = Store.YES, index = Index.YES)*]
	private String country;
	
	@ManyToMany(cascade = {
		CascadeType.MERGE, 
		CascadeType.DETACH, 
		CascadeType.PERSIST, 
		CascadeType.REFRESH
	})
	@JoinTable(name = "Author_Book", 
		joinColumns = 
			@JoinColumn(name = "authorFk", 
				referencedColumnName = "authorId"),
		inverseJoinColumns = 
			@JoinColumn(name = "bookFk", 
				referencedColumnName = "isbn"))
	[*@ContainedIn*]
	private Set<Book> books;
	
	//getters & setters ...
}
\end{lstlisting}
\noindent
As these annotations are defined in hibernate-search-engine, we can rely on all of them while designing the standalone version of Hibernate Search and all other modules depending on it.

\subsection{Using Hibernate Search's engine}

As already described earlier (\ref{problem_indexing_searching}), hibernate-search-engine is not intended to be used by application developers, but for other APIs to integrate with. Therefore there is no real public documentation available on how to use it and all following information had to be retrieved from tests in the hibernate-search-engine and hibernate-search-orm integration module source code.

\subsubsection{Starting the engine}
A Hibernate Search engine instance is represented by a \textbf{SearchIntegrator}. In order to obtain it, we first have to write a special configuration class that implements \textbf{org.hibernate.search.cfg.spi.SearchConfiguration}. An object of this class has then to be created and filled with all the configuration properties Hibernate Search requires. The minimum that has to be set for this to work map are the following properties:

\begin{enumerate}
	\item \textbf{hibernate.search.default.directory\_provider}: The two most common cases here are either "ram" or "filesystem". This decides where the index will be stored. A ram directory is only present in the system memory while the SearchIntegrator exists. A "filesystem" directory is persisted on the hard disk. For "filesystem" the additional property "hibernate.search.default.indexBase" has to be set to an appropriate path.
	
	\item \textbf{hibernate.search.lucene\_version}: This decides which Lucene version has to be used internally. The currently latest supported version is "4.10.4".
\end{enumerate}
\noindent
A complete list of the available settings can be found in the Hibernate Search documentation\footnote{Hibernate Search documentation, see~\cite{hibernate_search_doc}} (only some Hibernate ORM specific settings cannot be used). Our \textbf{StandaloneSearchConfiguration} (appendix listing \ref{lst:StandaloneSearchConfiguration.java}) defaults to "ram" and "4.10.4".
\\\\
Having this class in place, a \textbf{SearchIntegrator} can be obtained by a \textbf{SearchIntegratorBuilder} like this:
\\
\lstset{language=java}
\lstset{moredelim=[is][\bfseries]{[*}{*]}}
\begin{lstlisting}[frame=htrbl, caption={Starting up the engine}, label={lst:starting_up_engine.java}]
List<Class<?>> indexClasses = Arrays.asList(Book.class, Author.class);

SearchConfiguration searchConfiguration = 
	new StandaloneSearchConfiguration();
indexClasses.forEach( searchConfiguration::addClass );

//bootstrapping class for Hibernate Search
SearchIntegratorBuilder builder = new SearchIntegratorBuilder();

//we have to build an integrator here (the builder needs a 
//"base integrator" first before we can add index classes)
builder.configuration( searchConfiguration ).buildSearchIntegrator();

indexClasses.forEach( builder::addClass );

//starts the engine with all configuration properties set
SearchIntegrator searchIntegrator = builder.buildSearchIntegrator();

//use the integrator ...

//close it
searchIntegrator.close();
\end{lstlisting}

\subsubsection{Indexing, updating and deleting objects from the index}

Now that we know how a SearchIntegrator can be built, we can take a look at how we can control the index using the engine's features. 
\\\\
The engine does a lot of optimizations in the backend. This is the reason the specifics are hidden behind a \textbf{Worker} pattern. Such a worker batches operations by synchronizing upon the \textbf{org.hibernate.search.backend.TransactionContext} interface. Our implementation of this is simply called \textbf{Transaction} (appendix listing \ref{lst:Transaction.java}). The different index operations are represented by \textbf{Work} objects that contain the WorkType (INDEX, UPDATE, PURGE, etc.) and all necessary data to execute the individual task.
\\\\
Indexing objects with \textbf{WorkType.INDEX}:
\\
\lstset{language=java}
\begin{lstlisting}[frame=htrbl, caption={Indexing an object with the engine}, label={lst:indexing_object_native.java}]
Book book = ...;
Transaction tx = new Transaction();
Worker worker = searchIntegrator.getWorker();
worker.performWork( new Work( book, WorkType.INDEX ), tx );
tx.commit();
\end{lstlisting}
~\\
Updating objects with \textbf{WorkType.UPDATE}:
\\
\lstset{language=java}
\begin{lstlisting}[frame=htrbl, caption={Updating an object with the engine}, label={lst:updating_object_native.java}]
Book book = ...;
Transaction tx = new Transaction();
Worker worker = searchIntegrator.getWorker();
worker.performWork( new Work( book, WorkType.UPDATE ), tx );
tx.commit();
\end{lstlisting}
~\\
Deleting objects with \textbf{WorkType.PURGE}:
\\
\lstset{language=java}
\begin{lstlisting}[frame=htrbl, caption={Deleting an object by id with the engine}, label={lst:deleting_object_native.java}]
String isbn = ...;
Transaction tx = new Transaction();
Worker worker = searchIntegrator.getWorker();
worker.performWork( new Work( Book.class, isbn, WorkType.PURGE ), tx );
tx.commit();
\end{lstlisting}
~\\
This API doesn't have any "convenience" methods that wrap around the Transaction management if no batching is needed, nor does it have any wrapper utility for the Work object generation.

\subsubsection{Querying the index}

\lstset{language=java}
\begin{lstlisting}[frame=htrbl, caption={Querying the index with the engine}, label={lst:querying_natively.java}]
SearchIntegrator searchIntegrator = ...;

//find information about all the entities matching a given title
List<EntityInfo> entityInfos = 
	searchIntegrator.createHSQuery().luceneQuery( 
			searchIntegrator.buildQueryBuilder()
				.forEntity( Book.class )
				.get()
				.keyword()
				.onField( "title" )
				.matching( "searchString" )
				.createQuery()
		).targetedEntities(
			Collections.singletonList(
				Book.class
			)
		).projection(
			ProjectionConstants.ID
		).queryEntityInfos();

//extract info from the entityInfos
for(EntityInfo entityInfo : entityInfos) {
	String isbn = (String) entityInfo.getProjection()[0];
	//handle this information ...
}
\end{lstlisting}

\subsection{Standalone version of Hibernate Search}


\textcolor{red}{Architektur, Klassendiagramm, zusätzliche Features (DtoDescriptor)}

\begin{figure}[ht]
	\centering
	\includegraphics[scale=0.7]{images/standalone_min_architecture.pdf}
	\caption{Rough architecture of the standalone}
	\label{standalone_min_architecture}
\end{figure}

\subsubsection{Starting the standalone}

\lstset{language=java}
\lstset{moredelim=[is][\bfseries]{[*}{*]}}
\begin{lstlisting}[frame=htrbl, caption={Starting up the standalone}, label={lst:using_standalone.java}]
List<Class<?>> indexClasses = Arrays.asList(Book.class, Author.class);

SearchConfiguration searchConfiguration = 
		new StandaloneSearchConfiguration();
indexClasses.forEach( searchConfiguration::addClass );

StandaloneSearchFactory searchFactory = 
		StandaloneSearchFactoryFactory.
				createSearchFactory(
					searchConfiguration,
					indexClasses
				);
				
//use the searchfactory ...

//close it
searchFactory.close();
\end{lstlisting}

\subsubsection{Indexing, updating and deleting objects from the index}

\lstset{language=java}
\begin{lstlisting}[frame=htrbl, caption={Indexing an object with the standalone}, label={lst:indexing_object_native.java}]
Book book = ...;
searchFactory.index(book);
\end{lstlisting}

\lstset{language=java}
\begin{lstlisting}[frame=htrbl, caption={Updating an object with the standalone}, label={lst:updating_object_native.java}]
Book book = ...;
searchFactory.update(book);
\end{lstlisting}

\lstset{language=java}
\begin{lstlisting}[frame=htrbl, caption={Deleting an object by id with the standalone}, label={lst:deleting_object_native.java}]
String isbn = ...;
searchFactory.delete(Book.class, isbn);
\end{lstlisting}

\subsubsection{Querying the index}

\lstset{language=java}
\begin{lstlisting}[frame=htrbl, caption={Querying the index with the standalone}, label={lst:querying_natively.java}]
StandaloneSearchFactory searchFactory = ...;
EntityProvider entityProvider = ...;

List<Book> = searchFactory.createQuery(searchFactory.buildQueryBuilder()
				.forEntity(Book.class)
				.get()
				.keyword()
				.onField("title")
				.matching("searchString")
				.createQuery(), Book.class
			).query(
				entityProvider,
				Fetch.BATCH
			);
\end{lstlisting}

\subsection{Standalone integration with JPA interfaces}

\textcolor{red}{zusätzliche annotations, restriktionen!}

\pagebreak

\input{5_massindexing}

% !TeX spellcheck = en_GB
\subsection{The automatic index updating feature}

As already stated in \ref{automatic_indexing_problematic_intro}, the automatic index updating feature is a required for a reasonable Hibernate Search GenericJPA. As this is arguably the most complicated feature for GenericJPA, we will go into detail about how we are achieving it compared to the short introductions of the other features.

\subsubsection{Description of different implementations} \label{description_of_different_implementations}

There are several approaches to building an automatic index updating feature. While they are all different in the specifics, they can generally be separated into two categories: \textbf{synchronous} and \textbf{asynchronous}. Synchronous in this context means that the index is updated as soon as the newly changed data is persisted in the database without any real delay while in an asynchronous updating mechanism an arbitrary amount of time passes before the index is updated. While synchronous approaches are needed in some rare cases, fulltext search generally doesn't require a 100\% up-to-date at every point in time index as a search index generally is not the source of truth in an application (only the database contains the "truth").
\\\\
We will now work out a solution for both sync and async, while the async version will serve as a backup whenever the synchronized mechanism is not applicable.

\pagebreak

\paragraph{Synchronous approach}

For the synchronous approach we have two candidates: A system based on JPA callback events and another one that uses the native APIs of JPA providers. We start with the JPA callbacks and then go onto the native APIs.

\subparagraph{JPA events}

As we are trying to work with as little vendor specific APIs, JPA's callback events looks like a suitable candidate for listening to changes in entities.
\\\\
To listen for the JPA events we have two options: annotate the entities with callback methods or create a separate listener class. We will only take a look at the listener class since we don't want to have unnecessary methods in a possible user's entities. This class doesn't have to implement an interface, but has to have methods annotated with special annotations. The relevant ones are @PostPersist, @PostUpdate, @PostDelete (there are "pre-versions" available as well, but we focus on the post methods as they are more useful). What each specific annotation stands for is quite self-explanatory.
\\\\
Such a class generally looks like this:
\\
\lstset{language=java}
\lstset{moredelim=[is][\bfseries]{[*}{*]}}
\begin{lstlisting}[frame=htrbl, caption={Example JPA entity listener}, label={lst:jpa_entity_listener.java}]
public class EntityListener {

	@PostPersist
	public void persist(Object entity) {
		//handle the event
	}
	
	@PostUpdate
	public void update(Object entity) {
		//handle the event
	}
	
	@PostDelete
	public void delete(Object entity) {
		//handle the event
	}

}
\end{lstlisting}
\noindent
It is then applied with an annotation on the entity:
\\
\lstset{language=java}
\lstset{moredelim=[is][\bfseries]{[*}{*]}}
\begin{lstlisting}[frame=htrbl, caption={Using a JPA entity listener}, label={lst:using_jpa_entitylisteners.java}]
[*@EntityListeners( { EntityListener.class } )*]
public class Book {

	//...

}
\end{lstlisting}
\noindent
As the JPA provider creates the EnityListeners automatically, we have no access to them without injecting a reference to them in a static way. While this might cause some Classloader problems, it should be fine in most cases.
\\
\lstset{language=java}
\lstset{moredelim=[is][\bfseries]{[*}{*]}}
\begin{lstlisting}[frame=htrbl, caption={Injecting the EntityListener}, label={lst:jpa_entity_listener.java}]
public class EntityListener {

	public EntityListener() {
		// inject it somewhere
		// so we can access it in a static way
		EntityListenerRegistry.inject(this);
	}

	//...

}
\end{lstlisting}

\noindent
Even though these listeners seem to be the perfect fit as they would enable us to fully integrate only with JPA interfaces, they have two big issues as we find out after investigating further.
\\\\
Firstly, not all JPA providers seem to handle these events similar: For example Hibernate ORM doesn't propagate events from collection tables to the owning entity, while EclipseLink does (EclipseLink's behaviour would be needed from all providers).
\\\\
Secondly, we can see that the events are triggered on flush instead of commit. This is an issue if the changed data is not actually commited.
\\

\lstset{language=java}
\lstset{moredelim=[is][\bfseries]{[*}{*]}}
\begin{lstlisting}[frame=htrbl, caption={Event triggering on flush}, label={lst:flush_event.java}]
EntityManager em = ...;

em.getTransaction().begin();

Book book = em.find( Book.class "someIsbn" );
book.setTitle( "someNewTitle" );

// flushes, so we retrieve the Book with the changes from above
// => event is triggered
List<Book> allBooks = 
	em.createQuery( "SELECT b FROM Book b" ).getResultList();

// we have no way to get this event to revert the wrong index change
em.getTransaction().rollback();
\end{lstlisting}

\noindent
While it \textbf{might} be possible to somehow fix the flush issue, the bad support from JPA providers like Hibernate ORM renders this approach unusable until the JPA providers work the same way to some reasonable extent.

\pagebreak

\subparagraph{Native integration with JPA providers}

Almost every JPA provider has its own internal event system that is useful for cache invalidation and other tasks. These combined with hooks into the transaction management allow us to build a proper index updating system that works with transactions in mind (big improvement compared to the flush() issues of plain JPA)
\\\\
They generally have callbacks similar to these of the JPA events (no knowledge about database specifics is needed, Java types are used), but also provide additional information about the database session that caused the changes.
\\\\
By definition, these kind of integrations are not portable between JPA providers and require us to write different systems for all the JPA providers. But as the landscape for popular JPA providers probably only consists of Hibernate ORM, EclipseLink and OpenJPA, we can implement listeners for these and the others will have to rely on the async backup approach (as of the time of writing this, we have only implemented integrations for Hibernate ORM and EclipseLink).
\\\\
As this seems to be the only reasonable solution for a synchronous update system, we are using it for Hibernate Search GenericJPA even though it is no real native solution because of the JPA implementation dependent code.
\\\\
\textit{Note: we don't describe how these event systems are built in particular as they differ a lot in their APIs, but generally these are straightforward to use and describing the implementations would be unspectacular.}

\pagebreak

\paragraph{Asynchronous approach}
In contrary to the synchronous approach where we described two different versions, for the asynchronous version we only have one feasible solution available: A trigger based system.
\\\\
Triggers are "procedural code that is automatically executed in response to certain events on a particular table or view in a database" \footnote{Wikipedia on RDBMS triggers, see~\cite{triggers_wiki}}. While they are mostly "used for maintaining the integrity of the information on the database" \footnote{Wikipedia on RDBMS triggers, see~\cite{triggers_wiki}}, they are also useful for listening to events.
\\\\
Almost every RDBMS at least supports triggers on the three crucial events for event-listening: INSERT (CREATE), UPDATE, DELETE.
\\\\
In order to have triggers being useful for updating our Hibernate Search index, we have to get info about the events from the database back into our Java application. Since we cannot necessarily call Java code from our database (with the exception of some enterprise and in-memory databases), we have to write data about changes into auxiliary tables and then poll these regularly.
\\\\
One benefit of this approach is that by using polling from the tables and the - by definition transactional - triggers, we don't have to hook into transactions or deal with data that has not been committed, yet, in general. If we do things right, we can even improve indexing performance by this: We can query for the latest event for each entity only, so we don't use up an unnecessary amount of CPU-time we don't need, but still keep the index up-to-date.

\pagebreak

\subparagraph{Trigger architecture}

Triggers are generally created on tables. Since we want to use them for event-listening, we have to cover every table of the domain model that contains data indexed/stored in the index. This also includes all of the mapping tables between entities and all other secondary tables.
\\\\
The following figure shows the trigger architecture needed for our Author and Book example.

\begin{figure}[ht]
	\centering
	\includegraphics[scale=0.6]{images/Triggers_Schema.pdf}
	\caption{Triggers for the example project}
	\label{triggers_schema}
\end{figure}
\noindent
All three tables Author, Book and Author\_Book have three triggers registered on them (one for each event type). These triggers then fill up the update tables AuthorUpdates, BookUpdates and Author\_BookUpdates (these names are just for demonstrative purposes) with info about occurring events. We can see that these update tables host at least three things:

\begin{enumerate}
	\item \textbf{updateid primary key:} Update events have to be sortable by the order they occured. All Update tables share the same sequence of primary keys so that no key appears twice in all of these tables.
	\item \textbf{eventcase column:} This column contains a identifier for the cases INSERT, DELETE or UPDATE.
	\item \textbf{pseudo foreign key(s):} The relevant primary keys of the entities involved in the tables have to be stored in the Update tables as well. Note that they are not marked as real foreign keys as a DELETE event wouldn't work then (we can't have a reference to a non existent entity).
\end{enumerate}

\pagebreak

\subparagraph{Creating the tables} \label{creating_the_tables}

Since the creation of these tables requires a lot of work to be done, we have to automate it as good as possible. We do this by requiring additional annotations on the entities to map the required information for the update tables and then generating them out of it.
\\\\
These annotations contain at least the original table's name (UpdateInfo\#tableName) and the names \& types (IdColumn\#column \& IdColumn\#columnType) of the entity key columns. The name of the update table and the columns in it are then generally derived automatically from that.
\\\\
\textit{Note: The update tables are NO JPA entities, so we have to work with native SQL in the backend}
\\
\lstset{language=java}
\lstset{moredelim=[is][\bfseries]{[*}{*]}}
\begin{lstlisting}[frame=htrbl, caption={Book.java with Hibernate Search annotations}, label={lst:book.java_3}]
@Entity
@InIndex
@Table(name = "Book")
@Indexed
[*@UpdateInfo(
	tableName = "Book", 
	idInfos = @IdInfo(
		columns = @IdColumn(
			column = "isbn", 
			columnType = ColumnType.STRING
		)
	)
)*]
public class Book {

	// ... unchanged. 
	
	//mapping table events handled on Author side
	
	//getters & setters ...
}
\end{lstlisting}

\pagebreak

\lstset{language=java}
\lstset{moredelim=[is][\bfseries]{[*}{*]}}
\begin{lstlisting}[frame=htrbl, caption={Author.java with Hibernate Search annotations}, label={lst:author.java_3}]
@Entity
@InIndex
@Table(name = "Author")
[*@UpdateInfo(
	tableName = "Author", 
	idInfos = @IdInfo(
		columns = @IdColumn(
			column = "authorId", 
			columnType = ColumnType.LONG
		)
	)
)*]
public class Author {
	
	// ... unchanged.
	
	[*@UpdateInfo(tableName = "Author_Book", 
		idInfos = {
		@IdInfo(entity = Author.class, 
			columns = @IdColumn(
				column = "authorFk",
				columnType = ColumnType.LONG
			)
		),
		@IdInfo(entity = Book.class,
			columns = @IdColumn(
				column = "bookFk",
				columnType = ColumnType.STRING
			)
		)
	})*]
	private Set<Book> books;
	
	//getters & setters ...
}
\end{lstlisting}
\noindent
However, if the developer needs different names in the update tables, it is possible to manually set these properties. They can be found on the same level as the corresponding info for the original table is set.
\\\\
Options for multivalued keys and custom column types are also available as by default only singular valued keys of the column types corresponding to Java's Integer, Long and String are supported. While we don't go into detail how these expert features are used, information about how to use them can be found in the Javadoc of the annotations.
\\\\
Since database triggers and tables are not created the same on every RDBMS, we have to build an abstraction to get the necessary SQL code. This is done with the \textbf{TriggerSQLStringSource} interface. Its implementations return the specific SQL strings working on the corresponding RDBMS. As of this writing we have implementations for MySQL, PostgreSQL and HSQDLB. See table \ref{table:config_properties_jpasearchfactorycontroller} for information about using these.
\\\\
Whether and how the triggers and tables are generated at all can also be set, but with a configuration property on the SearchFactoryController as described in table  \ref{table:config_properties_jpasearchfactorycontroller}. If disabled, the user still has to provide the information about the update tables that should be used for updating with the annotations as described above.

\pagebreak

\subparagraph{Retrieving the events}
Now that we know how the events are stored in the update tables, we will now describe an efficient way to query the database for these entries.
\\\\
We only need the latest event for each entity (or combination of entities for mapping tables). The following SQL query is doing this for the table author\_bookupdates with standard SQL that should be working on every RDBMS.
\\
\lstset{language=sql}
\lstset{moredelim=[is][\bfseries]{[*}{*]}}
\begin{lstlisting}[frame=htrbl, caption={Querying for updates (Author\_Book)},
label={lst:querying_updates.sql}]
SELECT t1.updateidhsearch, t1.authorFkfk, t1.bookFkfk
FROM author_bookupdates t1
INNER JOIN
(
	/* select the most recent update */
	SELECT max(t2.updateidhsearch) updateid, 
		t2.authorFkfk, t2.bookFkfk
	FROM author_bookupdates t2
	GROUP BY t2.authorFkfk, t2.bookFkfk
) t3 on t1.updateidhsearch = t3.updateid
/* handle events that occured earlier first */
ORDER BY t1.updateidhsearch ASC;
\end{lstlisting}
\noindent
We run queries of this type for every update table with fixed delays (configurable, see table \ref{table:config_properties_jpasearchfactorycontroller}). Then, we scroll from the results
of these queries simultaneously while ordering by the updateids between the queries to make sure the events are definitely handled in the right order (see listing \ref{lst:MultiQueryAccess.java} in the appendix).
\\\\
This information is all we need to keep our index up-to-date. For the INSERT and UPDATE case we can just query the database for a new version and pass that to the engine. For the DELETE case we have to work directly on the index and have to enforce  \textbf{@IndexedEmbedded\#includeEmbeddedObjectId = true}. This is required so that we can determine the root entity in the index as its entry has to be updated additionally if the original entity is changed (A entity contained in one index can have its own index as well).
\\\\
After the index is updated accordingly, we run a delete query that deletes all update events
having an updateid lower than the last processed one for each table.
\\
\lstset{language=sql}
\lstset{moredelim=[is][\bfseries]{[*}{*]}}
\begin{lstlisting}[frame=htrbl, caption={Deleting handled updates (Author\_Book)},
label={lst:deleting_updates.sql}]
DELETE FROM author_bookupdates WHERE updateidhsearch < #last_handled_id#
\end{lstlisting}
\noindent
With these two types of queries for each update table we are able to keep the index up-to-date efficiently and also make sure that no event is handled twice.

\pagebreak

\subsubsection{Comparison of approaches}

We already discussed the differences of synchronous and asynchronous approaches in general earlier this chapter. The two chosen implementations differ in terms of extra work that has to be done to get them to work (user-friendliness for the developer) and features.

\paragraph{Additional work}
Since the native event system gets the proper information about changes from the vendor side, it doesn't require a lot information about the general structure of the domain model and tables in the database. For the Trigger based event system, that's a different story as it has to poll info about changes from the database. This is the reason the user has to add this information as we have seen in \ref{creating_the_tables}.
\\
\begin{figure}[ht]
	\centering
	\includegraphics[scale=0.6]{images/UpdateConsumer_Architecture.pdf}
	\caption{Hibernate Search GenericJPA update mechanisms}
	\label{updateconsumer_architecture}
\end{figure}

\paragraph{Features}
The native event system has the exact same updating behaviour as Hibernate Search ORM's update mechanism because it works on the same principles of using the existing event APIs. It just works for more ORM providers.
\\\\
With this similarity come two important drawbacks:
\begin{enumerate}
	\item It (the mechanism) only works with specifically supported JPA APIs
	\item Database changes coming from anything else than JPA APIs are not recognized. This includes native SQL queries from EntityManagers. This also means that the database can only be used by the JPA application and no other scripts, small programs etc. should have write access to the database.
\end{enumerate}
\noindent
These two drawbacks are non-existent with the trigger event system as it doesn't require any specific JPA implementation (1) and works on the database level (2).

\paragraph{Conclusion}
We can see that both event systems can be useful in different cases. This is the reason we use both in Hibernate Search GenericJPA. The following table summarizes the pros and cons once again:

\begin{table}[h] 
	\centering
	\begin{tabular}{|c|c|c|}
		\hline 
		Approach & Pros & Cons \\ 
		\hline 
		\specialcell{Native Event System} & 
		\specialcell{+ No additional work \\ needed by the developer} & 
		\specialcell{- Relies on different\\ implementation- \\ specific APIs \\ (only works with \\ specifically supported ones) \\
					- Changes from outside\\ of the JPA provider \\are not recognized \\ (e.g. native SQL access)} \\ 
		\hline
		\specialcell{Trigger Event System} & 
		\specialcell{+ Works with any JPA \\implementation \\ (even rarely used ones) \\
					+ Changes from outside\\ of the JPA provider \\ are recognized \\ (e.g. native SQL access)} & 
		\specialcell{- Additional work by \\the developer needed \\ (annotations)} \\ 
		\hline
	\end{tabular}
	\footnotesize \caption{Pros and Cons of the two update systems}
	\label{table:pros_and_cons_update_systems}
\end{table}

\pagebreak

\section{Using Hibernate Search GenericJPA}

Having described how Hibernate Search GenericJPA works and is designed we will now take a look how it can be used in our example project (\ref{example_project}). While have already explained this part by part in each chapter, the following is everything put together.
\\\\
\textit{Note: we are using the async updating mechanism here.}

\subsection{Dependencies}

The following example needs to have at least these dependencies on the classpath:

\begin{enumerate}
	\item{EclipseLink 2.5.0}
	\item{HSQLDB 2.3.3 (in memory database)}
	\item{Hibernate Search GenericJPA}
\end{enumerate}

\pagebreak

\subsection{Entities}
First, we have to update the Entity mappings in the Java classes. We add the \textbf{@Indexed, @DocumentId, @Field, @IndexedEmbedded, @ContainedIn} as known from the original Hibernate Search ORM (\ref{setting_up_example_project}). Using Hibernate Search GenericJPA then requires us also to add the @InIndex on every entity contained in the index (\ref{using_hsearch_genericjpa_index}) and because we are using the async updating mechanism here, we have to add information about how to create the update tables as well (\ref{creating_the_tables}).
\\\\
The resulting entities with the changes highlighted look like this:
\\
\lstset{language=java}
\lstset{moredelim=[is][\bfseries]{[*}{*]}}
\begin{lstlisting}[frame=htrbl, caption={Book.java complete}, label={lst:book.java_complete}]
@Entity
@Table(name = "Book")
[*@InIndex*]
[*@Indexed*]
[*@UpdateInfo(tableName = "Book",
	idInfos = @IdInfo(
		columns = @IdColumn(
			column = "isbn",
			columnType = ColumnType.STRING)))*]
public class Book {
	
	@Id
	[*@DocumentId*]
	@Column(name = "isbn")
	private String isbn;
	
	@Column(name = "title")
	[*@Field*]
	private String title;
	
	@Column(name = "genre")
	[*@Field*]
	private String genre;
	
	@Lob
	@Column(name = "summary")
	[*@Field*]
	private String summary;
	
	@ManyToMany(mappedBy = "books", cascade = {
		CascadeType.MERGE,
		CascadeType.DETACH,
		CascadeType.PERSIST,
		CascadeType.REFRESH
	})
	[*@IndexedEmbedded(includeEmbeddedObjectId = true)*]
	private Set<Author> authors;
	
	// getters & setters ...
	
}
\end{lstlisting}

\lstset{language=java}
\lstset{moredelim=[is][\bfseries]{[*}{*]}}
\begin{lstlisting}[frame=htrbl, caption={Author.java complete}, label={lst:author.java_complete}]
@Entity
@Table(name = "Author")
[*@InIndex*]
[*@UpdateInfo(tableName = "Author", 
	idInfos = @IdInfo(
		columns = @IdColumn(
			column = "authorId",
			columnType = ColumnType.LONG
	)
))*]
public class Author {
	
	@Id
	@GeneratedValue(strategy = GenerationType.AUTO)
	@Column(name = "authorId")
	[*@DocumentId*]
	private Long authorId;
	
	@Column(name = "firstName")
	[*@Field*]
	private String firstName;
	
	@Column(name = "lastName")
	[*@Field*]
	private String lastName;
	
	@Column(name = "country")
	[*@Field*]
	private String country;
	
	@ManyToMany(cascade = {
		CascadeType.MERGE,
		CascadeType.DETACH,
		CascadeType.PERSIST,
		CascadeType.REFRESH
	})
	@JoinTable(name = "Author_Book", 
		joinColumns = @JoinColumn(name = "authorFk", 
			referencedColumnName = "authorId"),
		inverseJoinColumns = @JoinColumn(name = "bookFk",
			referencedColumnName = "isbn"))
	[*@UpdateInfo(tableName = "Author_Book",
		idInfos = {
			@IdInfo(entity = Author.class,
				columns = @IdColumn(
				column = "authorFk",
				columnType = ColumnType.LONG)),
			@IdInfo(entity = Book.class,
				columns = @IdColumn(
				column = "bookFk",
				columnType = ColumnType.STRING))
	})*]
	[*@ContainedIn*]
	private Set<Book> books;
	
	// getters & setters ...
	
}
\end{lstlisting}


\pagebreak

\subsection{persistence.xml}

The persistence.xml file for our JPA based project is straightforward. As we are using an in-memory database with HSQLDB, settings for the schema creation and the user management are not important as the database is recreated at every restart.
\\

\lstset{language=xml}
\begin{lstlisting}[frame=htrbl, caption={persistence.xml complete}, label={lst:persistence.xml_complete}]
<persistence xmlns="http://java.sun.com/xml/ns/persistence"
xmlns:xsi="http://www.w3.org/2001/XMLSchema-instance"
xsi:schemaLocation="http://java.sun.com/xml/ns/persistence
	http://java.sun.com/xml/ns/persistence/persistence_2_0.xsd"
	version="2.0">

	<persistence-unit name="EclipseLink_HSQLDB"
		transaction-type="RESOURCE_LOCAL">
		<provider>
			org.eclipse.persistence.jpa.PersistenceProvider
		</provider>
		<class>*.*.Author</class>
		<class>*.*.Book</class>
		<properties>
			<property name="javax.persistence.jdbc.driver"
				value="org.hsqldb.jdbcDriver"/>
			<property name="javax.persistence.jdbc.url"
				value="jdbc:hsqldb:mem:test"/>
			<property name="javax.persistence.jdbc.user"
				value="user"/>
			<property name="javax.persistence.jdbc.password"
				value="password"/>
			<property name="eclipselink.ddl-generation"
				value="drop-and-create-tables"/>
			<property name="eclipselink.logging.level"
				value="INFO"/>
			<property name="eclipselink.ddl-
				generation.output-mode"
				value="both"/>
		</properties>
	</persistence-unit>
	
</persistence>
\end{lstlisting}

\pagebreak

\subsection{Complete usage example}

In the following listing we show the whole lifecycle of a Hibernate Search GenericJPA based application. The relevant code passages are commented in the code.
\\

\lstset{language=java}
\lstset{moredelim=[is][\bfseries]{[*}{*]}}
\begin{lstlisting}[frame=htrbl, caption={Complete usage}, label={lst:complete_usage.java}]
Properties properties = new Properties();

// use the async backend
properties.setProperty(
	"hibernate.search.searchfactory.type",
	"sql"
);

// we are using HSQLDB, so use the right TriggerSource
properties.setProperty(
	"hibernate.search.trigger.source",
	"org.hibernate.search.genericjpa.db." + 
		"events.triggers.HSQLDBTriggerSQLStringSource"
);

// start up the EntityManagerFactory (entry-point to JPA)
// and create one EntityManager
EntityManagerFactory emf = Persistence
	.createEntityManagerFactory( "EclipseLink_HSQLDB" );
EntityManager em = emf.createEntityManager();

// start up Hibernate Search GenericJPA
JPASearchFactoryController searchController = 
	Setup.createSearchFactoryController( emf, properties );

// persist entities in the database
em.getTransaction().begin();
Author author = ...;
Book book = ...;
book.setAuthor( author );
em.persist( em );
em.getTransaction().commit();

// we are using an async backend, so wait a bit
// for the updating mechanism to handle the
// persist (Exception not handled here)
Thread.sleep( 10_000 );

// create a FullTextEntityManager
FullTextEntityManager fem = searchController
	.getFullTextEntityManager( em );

// query for all Books having the title "searchString"
FullTextQuery fullTextQuery = fem.createFullTextQuery(
	fem.getSearchFactory().buildQueryBuilder()
		.forEntity( Book.class )
		.get()
		.keyword()
		.onField( "title" )
		.matching( "searchString" )
		.createQuery(), 
	Book.class);

List<Book> books = (List<Book>) fullTextQuery.getResultList();

//handle the books
System.out.println( books );

// close everything 
// (FullTextEntityManager is not closed because
// the EntityManager is closed)
em.close();
searchController.close();
emf.close();
\end{lstlisting}
\noindent
Note that we didn't put the code into a main method. This is due to the fact that in a real application all this code would obviously not be put into one single method:
\\\\
The startup process of Hibernate Search GenericJPA is generally put into an extra lifecycle helper that stores a reference to the JPASearchFactoryController in a global variable upon application startup similar to what is generally done with JPA's EntityManagerFactory (at least in Java SE applications). All Search related code then acquires the reference to the JPASearchFactoryController from the global variable and uses it similar to the above code. The lifecycle helper is also responsible for closing the JPASearchFactoryController when the application is shutting down.

\pagebreak


\onecolumn
% einfacher Zeilenabstand
\singlespacing
% Literaturliste soll im Inhaltsverzeichnis auftauchen
%\addcontentsline{toc}{section}{Literaturverzeichnis}
% Literaturverzeichnis anzeigen
%\renewcommand\refname{Literaturverzeichnis}
%\bibliography{Hauptdatei}

%% Index soll Stichwortverzeichnis heissen
% \newpage
% % Stichwortverzeichnis soll im Inhaltsverzeichnis auftauchen
% \addcontentsline{toc}{section}{Stichwortverzeichnis}
% \renewcommand{\indexname}{Stichwortverzeichnis}
% % Stichwortverzeichnis endgueltig anzeigen
% \printindex

\addcontentsline{toc}{section}{References}
%
% ---- Bibliography ----
%
\begin{thebibliography}{99}
	%
	\bibitem {wiki_jpa}
	Wikipedia
	\url{https://en.wikipedia.org/wiki/Java_Persistence_API}, 07/16/2015
	
	\bibitem{hibernate_search_homepage}
	Hibernate Search project homepage
	\url{http://hibernate.org/search/}, 07/26/2015
	
	\bibitem{hibernate_search_doc}
	Hibernate Search documentation
	\url{http://hibernate.org/search/documentation/}, 07/31/2015
	
	\bibitem{hibernate_search_doc_massindexer}
	Hibernate Search documentation (MassIndexer, v5.4)
	\url{https://docs.jboss.org/hibernate/search/5.4/reference/en-US/html_single/#search-batchindex-massindexer}, 08/05/2015
	
	\bibitem{triggers_wiki}
	Wikipedia on RDBMS triggers
	\url{https://en.wikipedia.org/wiki/Database_trigger}, 08/12/2015
	
	\bibitem{elasticsearch_java_api}
	ElasticSearch Java API
	\url{[https://www.elastic.co/guide/en/elasticsearch/client/java-api/current/index.html]}, 07/27/2015
	
	\bibitem{solr_java_api}
	Solr Java API
	\url{https://wiki.apache.org/solr/Solrj}, 07/27/2015
	
	\bibitem{object_oriented_programming_wiki}
	Wikipedia on Object Oriented Programming (OOP)
	\url{https://en.wikipedia.org/wiki/Object-oriented_programming}, 07/27/2015
	
	\bibitem{wikibooks_on_jpa}
	Wikibooks on Java Persistence
	\url{https://en.wikibooks.org/wiki/Java_Persistence/What_is_JPA\%3F}, 07/27/2015
	
	\bibitem{hibernate_ogm}
	Hibernate OGM project homepage
	\url{http://hibernate.org/ogm/}, 07/27/2015
	
	\bibitem{hibernate_orm}
	Hibernate ORM project homepage
	\url{http://hibernate.org/orm/}, 07/27/2015
	
	\bibitem{openjpa}
	OpenJPA project homepage
	\url{http://openjpa.apache.org/}, 07/27/2015
	
	\bibitem{sql_like_w3schools}
	w3schools on SQL LIKE
	\url{http://www.w3schools.com/sql/sql_like.asp}, 07/27/2015
	
	\bibitem{eclipselink}
	EclipseLink project homepage
	\url{http://www.eclipse.org/eclipselink/}, 07/27/2015
	
	\bibitem{hsearch_source_code_git}
	Hibernate Search GitHub repository
	\url{https://github.com/hibernate/hibernate-search}, 07/26/2015
	
	\bibitem{jdbc_oracle}
	Oracle JDBC overview
	\url{http://www.oracle.com/technetwork/java/javase/jdbc/index.html}, 07/27/2015
	
	\bibitem{oledb_ms}
	Documentation on how to use OleDb with .NET
	\url{https://msdn.microsoft.com/en-us/library/5ybdbtte(v=vs.71).aspx}, 07/27/2015
	
	\bibitem{xkcd_competing_standards_source}		
	xkcd \#927 on competing standards
	\url{https://xkcd.com/927/}, 07/26/2015
	
	\bibitem {wiki_java_ee}
	Java Platform, Enterprise Edition
	Wikipedia
	\url{https://en.wikipedia.org/wiki/Java_Platform,_Enterprise_Edition}, 07/16/2015
	
	\bibitem {lucene_apache_org}
	Lucene Website
	\url{https://lucene.apache.org/core/}, 07/16/2015
	
	\bibitem {lucene_basic_concepts}
	Lucene Tutorial
	\url{http://www.lucenetutorial.com/basic-concepts.html}, 07/20/2015
	
	
\end{thebibliography}

\onehalfspacing
% evtl. Anhang
\newpage
\addcontentsline{toc}{section}{Listings}
\fancyhead[L]{Listings} %Kopfzeile links
\section*{Listings}

\lstset{language=java}
\begin{lstlisting}[frame=htrbl, caption={the simple Transaction contract}, label={lst:Transaction.java}]
public class Transaction implements TransactionContext {

	private boolean progress = true;
	private List<Synchronization> syncs = new ArrayList<>();
	
	@Override
	public boolean isTransactionInProgress() {
		return this.progress;
	}
	
	@Override
	public Object getTransactionIdentifier() {
		return this;
	}
	
	@Override
	public void registerSynchronization(
		Synchronization synchronization ) {
		this.syncs.add( synchronization );
	}
	
	/**
	 * @throws IllegalStateException if already commited/rolledback
	 */
	public void commit() {
		if ( !this.progress ) {
			throw new IllegalStateException( 
			"can't commit - " + 
			"No Search Transaction is in Progress!" );
		}
		this.progress = false;
		this.syncs.forEach( Synchronization::beforeCompletion );
		
		for ( Synchronization sync : this.syncs ) {
			sync.afterCompletion( Status.STATUS_COMMITTED );
		}
	}
	
	/**
	 * @throws IllegalStateException if already commited/rolledback
 	 */
	public void rollback() {
		if ( !this.progress ) {
			throw new IllegalStateException( 
			"can't rollback - " + 
			"No Search Transaction is in Progress!" );
		}
		this.progress = false;
		this.syncs.forEach( Synchronization::beforeCompletion );
	
		for ( Synchronization sync : this.syncs ) {
			sync.afterCompletion( Status.STATUS_ROLLEDBACK );
		}
	}

}
\end{lstlisting}

\begin{lstlisting}[frame=htrbl, caption={StandaloneSearchConfiguration.java}, label={lst:StandaloneSearchConfiguration.java}]
/**
 * Manually defines the configuration. 
 * Classes and properties are the only implemented options at the moment.
 *
 * @author Martin Braun (adaption), Emmanuel Bernard
 */
public class StandaloneSearchConfiguration 
	extends SearchConfigurationBase 
	implements SearchConfiguration {

	private final Logger LOGGER = 
		Logger.getLogger( 
			StandaloneSearchConfiguration.class.getName() 
		);
		
	private final Map<String, Class<?>> classes;
	private final Properties properties;
	private final HashMap<Class<? extends Service>, Object> 
		providedServices;
	private final InstanceInitializer initializer;
	private SearchMapping programmaticMapping;
	private boolean transactionsExpected = true;
	private boolean indexMetadataComplete = true;
	private boolean idProvidedImplicit = false;
	private ClassLoaderService classLoaderService;
	private ReflectionManager reflectionManager;

	public StandaloneSearchConfiguration() {
		this( new Properties() );
	}

	public StandaloneSearchConfiguration(Properties properties) {
		this( 
			SubClassSupportInstanceInitializer.INSTANCE, 
			properties
		);
	}

	public StandaloneSearchConfiguration(InstanceInitializer init) {
		this( new Properties() );
	}

	public StandaloneSearchConfiguration(InstanceInitializer init, 
		Properties properties) {
		this.initializer = init;
		this.classes = new HashMap<>();
		this.properties = properties;
		// default values if nothing was explicitly set
		this.properties.computeIfAbsent(
			"hibernate.search.default.directory_provider", 
			(key) -> {
				LOGGER.info( 
				  "defaulting to RAM directory-provider" 
				);
			return "ram";
		});
		this.properties.computeIfAbsent(
			"hibernate.search.lucene_version", 
			(key) -> {
				LOGGER.info( 
					"defaulting to Lucene Version: " 
					+ Version.LUCENE_4_10_4.toString() 
				);
				return Version.LUCENE_4_10_4.toString();
		});
		this.reflectionManager = new JavaReflectionManager();
		this.providedServices = new HashMap<>();
		this.classLoaderService = new DefaultClassLoaderService();
	}

	public StandaloneSearchConfiguration addProperty(String key,
		String value) {
		properties.setProperty( key, value );
		return this;
	}

	public StandaloneSearchConfiguration addClass(Class<?> indexed) {
		classes.put( indexed.getName(), indexed );
		return this;
	}

	@Override
	public Iterator<Class<?>> getClassMappings() {
		return classes.values().iterator();
	}

	@Override
	public Class<?> getClassMapping(String name) {
		return classes.get( name );
	}

	@Override
	public String getProperty(String propertyName) {
		return properties.getProperty( propertyName );
	}

	@Override
	public Properties getProperties() {
		return properties;
	}

	@Override
	public ReflectionManager getReflectionManager() {
		return this.reflectionManager;
	}

	@Override
	public SearchMapping getProgrammaticMapping() {
		return programmaticMapping;
	}

	public StandaloneSearchConfiguration setProgrammaticMapping(
			SearchMapping programmaticMapping
		) {
		this.programmaticMapping = programmaticMapping;
		return this;
	}

	@Override
	public Map<Class<? extends Service>, Object> 
		getProvidedServices() {
		return providedServices;
	}

	public void addProvidedService(
			Class<? extends Service> serviceRole,
			Object service
		) {
		providedServices.put( serviceRole, service );
	}

	@Override
	public boolean isTransactionManagerExpected() {
		return this.transactionsExpected;
	}

	public void setTransactionsExpected(
			boolean transactionsExpected) {
		this.transactionsExpected = transactionsExpected;
	}

	@Override
	public InstanceInitializer getInstanceInitializer() {
		return initializer;
	}

	@Override
	public boolean isIndexMetadataComplete() {
		return indexMetadataComplete;
	}

	public void setIndexMetadataComplete(
		boolean indexMetadataComplete) {
		this.indexMetadataComplete = indexMetadataComplete;
	}

	@Override
	public boolean isIdProvidedImplicit() {
		return idProvidedImplicit;
	}

	public StandaloneSearchConfiguration 
		setIdProvidedImplicit(boolean idProvidedImplicit) {
		this.idProvidedImplicit = idProvidedImplicit;
		return this;
	}

	@Override
	public ClassLoaderService getClassLoaderService() {
		return classLoaderService;
	}

	public void setClassLoaderService(
		ClassLoaderService ) {
		this.classLoaderService = classLoaderService;
	}

}
\end{lstlisting}

\begin{lstlisting}[frame=htrbl, caption={BasicEntityProvider.java}, label={lst:BasicEntityProvider.java}]
public class BasicEntityProvider implements EntityProvider {

	private static final String QUERY_FORMAT = 
		"SELECT obj FROM %s obj " +
		"WHERE obj.%s IN :ids";
	private final EntityManager em;
	private final Map<Class<?>, String> idProperties;

	public BasicEntityProvider(EntityManager em,
		Map<Class<?>, String> idProperties) {
		this.em = em;
		this.idProperties = idProperties;
	}

	@Override
	public void close() throws IOException {
		this.em.close();
	}

	@Override
	public Object get(Class<?> entityClass, Object id,
		Map<String, String> hints) {
		return this.em.find( entityClass, id );
	}

	@SuppressWarnings({"rawtypes", "unchecked"})
	@Override
	public List getBatch(Class<?> entityClass, List<Object> ids,
		Map<String, String> hints) {
		List<Object> ret = new ArrayList<>( ids.size() );
		if ( ids.size() > 0 ) {
			String idProperty = 
				this.idProperties.get( entityClass );
			String queryString = 
				String.format(
					QUERY_FORMAT,
					this.em.getMetamodel()
						.entity( entityClass )
						.getName(),
					idProperty
		);
		Query query = this.em.createQuery( queryString );
		query.setParameter( "ids", ids );
			ret.addAll( query.getResultList() );
		}
		return ret;
	}
	
	public void clearEm() {
	this.em.clear();
	}

	public EntityManager getEm() {
	return this.em;
	}

}
\end{lstlisting}


% Eidesstattliche Erklärung
\addcontentsline{toc}{section}{Eidesstattliche Erklärung}
\include{erklaerung}

% leere Abschlussseite
\newpage
\thispagestyle{empty} % erzeugt Seite ohne Kopf- / Fusszeile
\section*{ }

\end{document}