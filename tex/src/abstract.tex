% !TeX spellcheck = en_GB
\section*{Abstract}

Fulltext search engines are a powerful tool to improve query results in applications where relational databases don't suffice. However, they don't integrate well with the widely spread concept of object relationship mappers (ORM, in Java predominantly represented by the standard JPA) in the object oriented programming world.
\\\\
This is where Hibernate Search comes into use for Java developers: It combines JPA and fulltext search by being the intermediary between Hibernate ORM and a Lucene based fulltext index. It has one problem though: Hibernate Search only works with Hibernate ORM but not with other JPA-conform providers even though it is possible to support these. In this thesis we will show how such a generic version can be accomplished.
\\\\
After discussing the methods we use, we give an explanation why a generic Hibernate Search is a desirable solution for JPA developers. Creating it is challenging as we have to build a standalone version of Hibernate Search's internal engine first and then integrate it with JPA together with an automatic index updating mechanism. We solve these challenges and give a usage example of the completed generic version. Finally we discuss the current development state of the generic version and give an outlook on the planned merging process with the original Hibernate Search.

\pagebreak

\section*{Zusammenfassung}
Volltextsuchengines sind ein wertvolles Werkzeug um Suchergebnisse in Anwendungen zu verbessern, wenn relationale Datenbanken nicht ausreichen. Diese Engines sind jedoch nicht gut mit dem in der objekt-orientierten Programmierungs-Welt weit verbreiteten Konzept der Objekt-Relationalen Mapper (ORM, in Java vor allem durch den Standard JPA repräsentiert) integriert. 
\\\\
Für Java Entwickler bietet hier Hibernate Search eine Abhilfe: Es kombiniert JPA und Volltextsuche und stellt die Schnittstelle zwischen Hibernate ORM und einem Lucene basierten Volltextindex dar. Es hat aber ein Problem: Hibernate Search funktioniert nur in Kombination mit Hibernate ORM aber nicht mit anderen JPA konformen Providern, obwohl es möglich wäre diese zu unterstützen. In dieser Thesis wird daher gezeigt, wie eine solche generische Version realisiert werden kann.
\\\\
Nachdem die benutzten Methoden erklärt wurden, wird eine Begründung dafür gegeben, warum Hibernate Search eine wünschenswerte Lösung für JPA Entwickler ist. Diese zu entwickeln ist eine Herausforderung, da wir zuerst eine Standalone Version von Hibernate Search's interner Engine bauen müssen um diese danach in eine JPA Version zusammen mit einem automatischen Index Updating Mechanismus integrieren zu integrieren. Wir zeigen, wie diese Probleme gelöst werden und erklären die Benutzung anhand eines Beispiels. Zuletzt gehen wir auf den aktuellen Entwicklungsstand der generischen Version ein und geben einen Ausblick auf den geplanten Merge-Prozess mit dem originalen Hibernate Search.

