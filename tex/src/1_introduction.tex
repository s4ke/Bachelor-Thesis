\section{Introduction}\label{einleitung}

\subsection{An overview of different database paradigmas}

When it comes to persisting data in applications, nowadays there are a lot of different
paradigmas around that one has to choose from. Following is a short overview over the 
most common ones.

\subsubsection{relational databases}

\subsubsection{NoSQL databases}

\subsubsection{Object databases}

\subsection{Object-relational impedance mismatch}

While the NoSQL approach is undeniably rising in popularity, the demand for
relational solutions is still unmatched as it has been used and proven in practice
for many years now.
\\\\
Nowadays, many popular languages like Java, C\#, etc. are object-oriented.
While SQL solutions for querying relational databases exist for these languages, there is a big
discrepancy between the relational and the object-oriented paradigm. \footnote{Wikipedia, see~\cite{wiki_object_mismatch}}
\\\\
This is where Object Relational Mappers (ORM) come into use. They map tables to entities (classes) and
enable users to write queries against classes instead of tables. This is especially useful if used
in big software products as not all programmers have to know the exact details of the underlying database. The database system could even be completely replaced for another with the business logic still being in place.

\subsubsection{JPA}

\subsubsection{fulltext search}

\subsubsection{compatibility of Hibernate Search with JPA}

\subsubsection{aims of this thesis}
