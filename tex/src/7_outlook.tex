% !TeX spellcheck = en_GB
\section{Outlook}\label{outlook}

In this thesis we described how we can integrate Hibernate Search with JPA conform ORM implementations. We started by building a standalone integration of hibernate-search-engine, then integrated it with JPA and finally created an automatic index updating mechanism. All challenges described in \ref{Challenges} have been resolved.
\\\\
During the process of designing and writing the code for Hibernate Search GenericJPA we tried to be as compatible with the orginal Hibernate Search API as possible. While one reason for this is to make the switch easier for developers that want to try it out, the biggest reason is that the ultimate goal for this project is to be merged into the original Hibernate Search codebase.
\\\\
\fixme{more criticism, what has to be improved? -> the event system?}
This is the reason this project has to be looked as a proof of concept even though the code as it can be found on GitHub \footnote{Hibernate Search GenericJPA GitHub repository, see~\cite{hibernate_genericjpa_github}} can already be used in real applications.
\\\\
The first steps of the merging process have already been discussed with the Hibernate Search development team and work on the merging process is to be started in early October 2015. This comes exactly at the right moment as the Hibernate Search team is planning API changes in the near future \footnote{Hibernate Search roadmap, see~\cite{hibernate_search_roadmap}} as some interfaces have to be altered as we have seen in \ref{integration_jpa}.
\\\\
As soon as the generic version is part of Hibernate Search and is fully compatible with its API, Hibernate Search can be looked at as a standard for fulltext search in JPA. Having such a standard would be quite beneficial for the ever changing JPA world as smaller JPA providers could have a better chance at getting a bigger user base. And competition has never been a bad thing.

\fixme{mention, that everything in here is tested and proven to work. -> methods chapter}