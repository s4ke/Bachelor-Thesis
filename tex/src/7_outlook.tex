% !TeX spellcheck = en_GB

~
\pagebreak

\section{Outlook}\label{outlook}

In this thesis we described how we can integrate Hibernate Search with JPA conform ORM implementations. We started by building a standalone integration of hibernate-search-engine, then integrated it with JPA and finally created an automatic index updating mechanism. All challenges described in chapter \ref{Challenges} have been resolved.
\\\\
The only feature needing some extra work is probably the generic updating mechanism with database triggers. At the moment the developer has to specify additional annotations containing information about the update tables by hand. As mentioned in chapter \ref{creating_the_tables}, at least some of the information is known to be able to be retrieved directly from JPA annotations. These mechanisms are not included in this thesis but can be added in a future version.
\\\\
During the process of designing and writing the code for Hibernate Search GenericJPA we tried to be as compatible with the orginal Hibernate Search API as possible. While one reason for this is to make the switch easier for developers that want to try it out, the biggest one is that the ultimate goal for this project is to be merged into the original Hibernate Search codebase even though we haven't mentioned this in the beginning.
\\\\
This is also why this project has to be looked as a proof of concept even though the code as it can be found on GitHub \footnote{Hibernate Search GenericJPA GitHub repository, see~\cite{hibernate_genericjpa_github}} can already be used in real applications. In fact as described in chapter \ref{Methods} every relevant part of Hibernate Search GenericJPA has been extensively tested in single feature-tests and integration-tests and can therefore be considered stable. 
\\\\
The first steps of the merging process have already been discussed with the Hibernate Search development team and work on the merging process is to be started in November 2015. This comes exactly at the right moment as the Hibernate Search team is planning API changes in the near future and \footnote{Hibernate Search roadmap, see~\cite{hibernate_search_roadmap}} some interfaces have to be altered (as seen in chapter \ref{integration_jpa}) in order to support generic JPA.
\\\\
As soon as the generic version is part of Hibernate Search and is fully compatible with its API, Hibernate Search can be looked at as a de-facto standard for fulltext search in JPA. Having such a standard would be quite beneficial for the ever changing JPA world as smaller JPA providers could have a better chance at getting a bigger user base, which is good for research and innovation.

\pagebreak
~