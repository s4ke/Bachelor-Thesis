\section*{Listings}

Following are some interesting classes referenced in the thesis that were too long to fit into the text.
\\

\noindent
\textbf{Transaction}:\\
This class is the simple Transaction representation used to control index changes. It is not intended to be similar to a RDBMS transaction, but is merely a batch context with simple commit and rollback features.
\\
\lstset{language=java}
\begin{lstlisting}[frame=htrbl, caption={the simple Transaction contract}, label={lst:Transaction.java}]
public class Transaction implements TransactionContext {

	private boolean progress = true;
	private List<Synchronization> syncs = new ArrayList<>();
	
	@Override
	public boolean isTransactionInProgress() {
		return this.progress;
	}
	
	@Override
	public Object getTransactionIdentifier() {
		return this;
	}
	
	@Override
	public void registerSynchronization(
		Synchronization synchronization ) {
		this.syncs.add( synchronization );
	}
	
	/**
	 * @throws IllegalStateException if already commited/rolledback
	 */
	public void commit() {
		if ( !this.progress ) {
			throw new IllegalStateException( 
			"can't commit - " + 
			"No Search Transaction is in Progress!" );
		}
		this.progress = false;
		this.syncs.forEach( Synchronization::beforeCompletion );
		
		for ( Synchronization sync : this.syncs ) {
			sync.afterCompletion( Status.STATUS_COMMITTED );
		}
	}
	
	/**
	 * @throws IllegalStateException if already commited/rolledback
 	 */
	public void rollback() {
		if ( !this.progress ) {
			throw new IllegalStateException( 
			"can't rollback - " + 
			"No Search Transaction is in Progress!" );
		}
		this.progress = false;
		this.syncs.forEach( Synchronization::beforeCompletion );
	
		for ( Synchronization sync : this.syncs ) {
			sync.afterCompletion( Status.STATUS_ROLLEDBACK );
		}
	}

}
\end{lstlisting}

\pagebreak

\noindent
\textbf{StandaloneSearchConfiguration}:\\
hibernate-search-engine requires an object implementing the SearchConfiguration interface. StandaloneSearchConfiguration is the basic implementation of this used in our standalone version of Hibernate Search.
\\
\lstset{language=java}
\begin{lstlisting}[frame=htrbl, caption={StandaloneSearchConfiguration.java}, label={lst:StandaloneSearchConfiguration.java}]
/**
 * Manually defines the configuration. 
 * Classes and properties are the only implemented options at the moment.
 *
 * @author Martin Braun (adaption), Emmanuel Bernard
 */
public class StandaloneSearchConfiguration 
	extends SearchConfigurationBase 
	implements SearchConfiguration {

	private final Logger LOGGER = 
		Logger.getLogger( 
			StandaloneSearchConfiguration.class.getName() 
		);
		
	private final Map<String, Class<?>> classes;
	private final Properties properties;
	private final HashMap<Class<? extends Service>, Object> 
		providedServices;
	private final InstanceInitializer initializer;
	private SearchMapping programmaticMapping;
	private boolean transactionsExpected = true;
	private boolean indexMetadataComplete = true;
	private boolean idProvidedImplicit = false;
	private ClassLoaderService classLoaderService;
	private ReflectionManager reflectionManager;

	public StandaloneSearchConfiguration() {
		this( new Properties() );
	}

	public StandaloneSearchConfiguration(Properties properties) {
		this( 
			SubClassSupportInstanceInitializer.INSTANCE, 
			properties
		);
	}

	public StandaloneSearchConfiguration(InstanceInitializer init) {
		this( new Properties() );
	}

	public StandaloneSearchConfiguration(InstanceInitializer init, 
		Properties properties) {
		this.initializer = init;
		this.classes = new HashMap<>();
		this.properties = properties;
		// default values if nothing was explicitly set
		this.properties.computeIfAbsent(
			"hibernate.search.default.directory_provider", 
			(key) -> {
				LOGGER.info( 
				  "defaulting to RAM directory-provider" 
				);
			return "ram";
		});
		this.properties.computeIfAbsent(
			"hibernate.search.lucene_version", 
			(key) -> {
				LOGGER.info( 
					"defaulting to Lucene Version: " 
					+ Version.LUCENE_5_2_1.toString() 
				);
				return Version.LUCENE_5_2_1.toString();
		});
		this.reflectionManager = new JavaReflectionManager();
		this.providedServices = new HashMap<>();
		this.classLoaderService = new DefaultClassLoaderService();
	}

	public StandaloneSearchConfiguration addProperty(String key,
		String value) {
		properties.setProperty( key, value );
		return this;
	}

	public StandaloneSearchConfiguration addClass(Class<?> indexed) {
		classes.put( indexed.getName(), indexed );
		return this;
	}

	@Override
	public Iterator<Class<?>> getClassMappings() {
		return classes.values().iterator();
	}

	@Override
	public Class<?> getClassMapping(String name) {
		return classes.get( name );
	}

	@Override
	public String getProperty(String propertyName) {
		return properties.getProperty( propertyName );
	}

	@Override
	public Properties getProperties() {
		return properties;
	}

	@Override
	public ReflectionManager getReflectionManager() {
		return this.reflectionManager;
	}

	@Override
	public SearchMapping getProgrammaticMapping() {
		return programmaticMapping;
	}

	public StandaloneSearchConfiguration setProgrammaticMapping(
			SearchMapping programmaticMapping
		) {
		this.programmaticMapping = programmaticMapping;
		return this;
	}

	@Override
	public Map<Class<? extends Service>, Object> 
		getProvidedServices() {
		return providedServices;
	}

	public void addProvidedService(
			Class<? extends Service> serviceRole,
			Object service
		) {
		providedServices.put( serviceRole, service );
	}

	@Override
	public boolean isTransactionManagerExpected() {
		return this.transactionsExpected;
	}

	public void setTransactionsExpected(
			boolean transactionsExpected) {
		this.transactionsExpected = transactionsExpected;
	}

	@Override
	public InstanceInitializer getInstanceInitializer() {
		return initializer;
	}

	@Override
	public boolean isIndexMetadataComplete() {
		return indexMetadataComplete;
	}

	public void setIndexMetadataComplete(
		boolean indexMetadataComplete) {
		this.indexMetadataComplete = indexMetadataComplete;
	}

	@Override
	public boolean isIdProvidedImplicit() {
		return idProvidedImplicit;
	}

	public StandaloneSearchConfiguration 
		setIdProvidedImplicit(boolean idProvidedImplicit) {
		this.idProvidedImplicit = idProvidedImplicit;
		return this;
	}

	@Override
	public ClassLoaderService getClassLoaderService() {
		return classLoaderService;
	}

	public void setClassLoaderService(
		ClassLoaderService ) {
		this.classLoaderService = classLoaderService;
	}

}
\end{lstlisting}

\pagebreak

\noindent
\textbf{BasicEntityProvider}:\\
This is the basic implementation of the EntityProvider interface which is used to abstract the database access in the standalone version. It uses a JPA EntityManager to accomplish this.
\\
\lstset{language=java}
\begin{lstlisting}[frame=htrbl, caption={BasicEntityProvider.java}, label={lst:BasicEntityProvider.java}]
public class BasicEntityProvider implements EntityProvider {

	private static final String QUERY_FORMAT = 
		"SELECT obj FROM %s obj " +
		"WHERE obj.%s IN :ids";
	private final EntityManager em;
	private final Map<Class<?>, String> idProperties;

	public BasicEntityProvider(EntityManager em,
		Map<Class<?>, String> idProperties) {
		this.em = em;
		this.idProperties = idProperties;
	}

	@Override
	public void close() throws IOException {
		this.em.close();
	}

	@Override
	public Object get(Class<?> entityClass, Object id,
		Map<String, String> hints) {
		return this.em.find( entityClass, id );
	}

	@SuppressWarnings({"rawtypes", "unchecked"})
	@Override
	public List getBatch(Class<?> entityClass, List<Object> ids,
		Map<String, String> hints) {
		List<Object> ret = new ArrayList<>( ids.size() );
		if ( ids.size() > 0 ) {
			String idProperty = 
				this.idProperties.get( entityClass );
			String queryString = 
				String.format(
					QUERY_FORMAT,
					this.em.getMetamodel()
						.entity( entityClass )
						.getName(),
					idProperty
		);
		Query query = this.em.createQuery( queryString );
		query.setParameter( "ids", ids );
			ret.addAll( query.getResultList() );
		}
		return ret;
	}
	
	public void clearEm() {
		this.em.clear();
	}

	public EntityManager getEm() {
		return this.em;
	}

}
\end{lstlisting}

\pagebreak

\noindent
\textbf{Obtaining the idProperties}:\\
This code snippet shows how the idProperties map needed for the instantiation of a BasicEntityProvider can be obtained. This mechanism is used on some other places of Hibernate Search GenericJPA as well.
\\
\lstset{language=java}
\begin{lstlisting}[frame=htrbl, caption={Obtaining idProperties}, label={lst:idProperties.java}]
SearchConfiguration config = ...;

MetadataProvider metadataProvider = 
	MetadataUtil.getDummyMetadataProvider( config );
MetadataRehasher rehasher = new MetadataRehasher();

List<RehashedTypeMetadata> rehashedTypeMetadatas = new ArrayList<>();
for ( Class<?> indexRootType : this.getIndexRootTypes() ) {
	RehashedTypeMetadata rehashed = 
		rehasher.rehash( 
			metadataProvider
				.getTypeMetadataFor( indexRootType ) 
		);
	rehashedTypeMetadatas.add( rehashed );
}

Map<Class<?>, String> idProperties = 
	MetadataUtil.calculateIdProperties( rehashedTypeMetadatas );
\end{lstlisting}

\pagebreak

\noindent
\textbf{MultiQueryAccess}:\\
This is the utility class used to scroll results from multiple queries at once while retrieving the events from the database in the asynchronous approach.
\\
\lstset{language=java}
\begin{lstlisting}[frame=htrbl, caption={MultiQueryAccess.java}, label={lst:MultiQueryAccess.java}]
/**
 * Utility class that allows you to access multiple JPA queries at once.
 * Data is retrieved from the database in batches
 * and ordered by a given comparator.
 * No need for messy Unions on the database level! <br>
 * <br>
 * This is particularly useful if you scroll all the data 
 * from the database incrementally and if you can 
 * compare in Code.
 *
 * @author Martin
 */
public class MultiQueryAccess {

	private final Map<String, Long> currentCountMap;
	private final Map<String, Query> queryMap;
	private final Comparator<ObjectIdentifierWrapper> comparator;
	private final int batchSize;
	
	private final Map<String, Long> currentPosition;
	private final Map<String, LinkedList<Object>> values;
	
	private Object scheduled;
	private String identifier;


	public MultiQueryAccess(
		Map<String, Long> countMap,
		Map<String, Query> queryMap,
		Comparator<ObjectIdentifierWrapper> comparator,
		int batchSize) {
		if ( countMap.size() != queryMap.size() ) {
			throw new IllegalArgumentException( 
				"countMap.size() must be equal " + 
					"to queryMap.size()" );
		}
		this.currentCountMap = countMap;
		this.queryMap = queryMap;
		this.comparator = comparator;
		this.batchSize = batchSize;
		this.currentPosition = new HashMap<>();
		this.values = new HashMap<>();
		for ( String ident : queryMap.keySet() ) {
			this.values.put( ident, new LinkedList<>() );
			this.currentPosition.put( ident, 0L );
		}
	}

	private static int toInt(Long l) {
		return (int) (long) l;
	}

	/**
	 * increments the value to be returned by {@link #get()}
	 *
	 * @return true if there is a value left to be visited 
	 *	in the database
	 */
	public boolean next() {
	
	/*
	 *
	 *
	 * indentation broken to make this readable
	 *
	 *
	 */
	
this.scheduled = null;
this.identifier = null;
List<ObjectIdentifierWrapper> tmp =
	new ArrayList<>( this.queryMap.size() );

for ( Map.Entry<String, Query> entry : this.queryMap.entrySet() ) {
	String identifier = entry.getKey();
	Query query = entry.getValue();
	if ( !this.currentCountMap.get( identifier ).equals( 0L ) ) {
		if ( this.values.get( identifier ).size() == 0 ) {
			// the last batch is empty. get a new one
			Long processed = 
				this.currentPosition.get( identifier );
			// yay JPA...
			query.setFirstResult( toInt( processed ) );
			query.setMaxResults( this.batchSize );
			@SuppressWarnings("unchecked")
			List<Object> list = query.getResultList();
			this.values.get( identifier ).addAll( list );
		}
		Object val = this.values.get( identifier ).getFirst();
		tmp.add( new ObjectIdentifierWrapper( val, identifier ) );
	}
}
tmp.sort( this.comparator );
if ( tmp.size() > 0 ) {
	ObjectIdentifierWrapper arr = tmp.get( 0 );
	this.scheduled = arr.object;
	this.identifier = arr.identifier;
	this.values.get( this.identifier ).pop();
	Long currentPosition = this.currentPosition.get( arr.identifier );
	Long newCurrentPosition = 
		this.currentPosition
			.computeIfPresent( arr.identifier, 
				(clazz, old) -> old + 1 );
	if ( Math.abs( newCurrentPosition - currentPosition ) != 1L ) {
		throw new AssertionFailure( 
			"the new currentPosition count " + 
			"should be exactly 1 " +
			"greater than the old one" );
	}
	Long count = this.currentCountMap.get( arr.identifier );
	Long newCount = this.currentCountMap.computeIfPresent(
		arr.identifier, (clazz, old) -> old - 1
	);
	if ( Math.abs( count - newCount ) != 1L ) {
		throw new AssertionFailure( 
			"the new old remaining count " + 
				should be exactly 1 " +
				"greater than the new one" );
	}
}
return this.scheduled != null;
}

	/**
	* @return the current value
	*/
	public Object get() {
		if ( this.scheduled == null ) {
			throw new IllegalStateException(
				"either empty or next() has " + 
					"not been called" );
		}
		return this.scheduled;
	}

	/**
	* @return the identifier of the current value
	*/
	public String identifier() {
		if ( this.identifier == null ) {
			throw new IllegalStateException( 
				"either empty or next() has " + 
					"not been called" );
		}
		return this.identifier;
	}

	public static class ObjectIdentifierWrapper {
	
		public final Object object;
		public final String identifier;
		
		public ObjectIdentifierWrapper(Object object,
			String identifier) {
			this.object = object;
			this.identifier = identifier;
		}
	
	}

}
\end{lstlisting}

\pagebreak

~